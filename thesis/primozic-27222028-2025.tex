\documentclass[mat2, tisk]{fmfdelo}
% \documentclass[fin2, tisk]{fmfdelo}
% \documentclass[isrm2, tisk]{fmfdelo}
% \documentclass[ped, tisk]{fmfdelo}
% Če pobrišete možnost tisk, bodo povezave obarvane,
% na začetku pa ne bo praznih strani po naslovu, …
%%%%%%%%%%%%%%%%%%%%%%%

%%%%%%%%%%%%%%%%%%%%%
%%%%%%%%%%%%%%%%%%%%%%%%%%%%%%%%%%%%%%%%%%%%%%%%%%%%%%%%%%%%%%%%%%%%%%%%%%%%%%%
% METAPODATKI
%%%%%%%%%%%%%%%%%%%%%%%%%%%%%%%%%%%%%%%%%%%%%%%%%%%%%%%%%%%%%%%%%%%%%%%%%%%%%%%

% - vaše ime
\avtor{Urh Primožič}

% - naslov dela v slovenščini
\naslov{Iskanje morfizmov z gradientnim spustom}

% - naslov dela v angleščini
\title{Finding morphisms with gradient
descent}

% - ime mentorja/mentorice s polnim nazivom:
%   - doc.~dr.~Ime Priimek
%   - izr.~prof.~dr.~Ime Priimek
%   - prof.~dr.~Ime Priimek
%   za druge variante uporabite ustrezne ukaze
\mentor{prof.~dr.~Ljupčo Todorovski}
\somentor{doc.~dr.~Urban Jezernik}
% \mentorica{...}
% \somentorica{...}
% \mentorja{...}{...}
% \somentorja{...}{...}
% \mentorici{...}{...}
% \somentorici{...}{...}

% - leto magisterija
\letnica{2025}

% - povzetek v slovenščini
%   V povzetku na kratko opišite vsebinske rezultate dela. Sem ne sodi razlaga
%   organizacije dela, torej v katerem razdelku je kaj, pač pa le opis vsebine.
\povzetek{Tukaj napišemo povzetek vsebine. Sem sodi razlaga vsebine
  in ne opis tega, kako je delo organizirano.

}

% - povzetek v angleščini
\abstract{An abstract of the work is written here. This includes a
  short description of
the content and not the structure of your work.}

% - klasifikacijske oznake, ločene z vejicami
%   Oznake, ki opisujejo področje dela,\textbf{} so dostopne na
% strani https://www.ams.org/msc/

% 15A69  Multilinear algebra: matrix/tensor parameterization
% 20C30  Representations of symmetric and related finite groups
% 20C99  Group representation aspects not covered elsewhere
% 68T05  Machine learning: gradient-based learning systems
% 65K10  Numerical optimization: gradient descent and variational methods
% 68T09  AI methods in symbolic group computation

\klasifikacija{65K10, 68T05, 20C99, 68T09}

% - ključne besede, ki nastopajo v delu, ločene s \sep
\kljucnebesede{gradientni spust\sep upodobitve\sep izomorfizmi
  grafov\sep delovanja\sep grafovske nevronske mreže
  % \texorpdfstring{$C^*$}{C*}-algebre
}

% - angleški prevod ključnih besed
\keywords{
  gradient descent\sep representations\sep graph isomorphisms\sep
  actions\sep graph neural networks
}

% - neobvezna zahvala
\zahvala{
  Neobvezno.
  Zahvaljujem se \dots
}

% - ime datoteke z viri (vključno s končnico .bib), če uporabljate BibTeX
\literatura{literatura.bib}

%%%%%%%%%%%%%%%%%%%%%%%%%%%%%%%%%%%%%%%%%%%%%%%%%%%%%%%%%%%%%%%%%%%%%%%%%%%%%%%
% DODATNE DEFINICIJE
%%%%%%%%%%%%%%%%%%%%%%%%%%%%%%%%%%%%%%%%%%%%%%%%%%%%%%%%%%%%%%%%%%%%%%%%%%%%%%%

% naložite dodatne pakete, ki jih potrebujete
\usepackage{pdfpages}
\usepackage{units}        % fizikalne enote kot \unit[12]{kg} s
% polovico nedeljivega presledka, glej primer v kodi
\usepackage{graphicx}     % za slike
\usepackage{subcaption}
% \usepackage{pgffor}
\usepackage{tikz}
\usepackage{etoolbox}
\usepackage{xparse}
\usepackage{tabularx}
\usepackage{mathdots}
% \usepackage{tikz}
% VEČ ZANIMIVIH PAKETOV
% \usepackage{array}      % več možnosti za tabele
% \usepackage[list=true,listformat=simple]{subcaption}  % več kot ena
% slika na figure, omogoči slika 1a, slika 1b
% \usepackage[all]{xy}    % diagrami
% \usepackage{doi}        % za clickable DOI entrye v bibliografiji
% \usepackage{enumerate}     % več možnosti za sezname

% Za barvanje source kode
% \usepackage{minted}
% \renewcommand\listingscaption{Program}

% Za pisanje psevdokode
% \usepackage{algpseudocode}  % za psevdokodo
% \usepackage{algorithm}
% \floatname{algorithm}{Algoritem}
% \renewcommand{\listalgorithmname}{Kazalo algoritmov}

% deklarirajte vse matematične operatorje, da jih bo LaTeX pravilno stavil
% \DeclareMathOperator{\...}{...}

% vstavite svoje definicije ...
%%%%%%%%%%%%%%%%%%%
% MOJE DEFINICIJE
%%%%%%%%%%%%%
% \fortable{n}{begin}{end}{n_columns}{column} naredi tabelo
\newcommand{\TODO}[1]{{\color{blue} TODO: #1}}
\newcommand{\R}{\mathbb R}
\newcommand{\N}{\mathbb N}
\newcommand{\Z}{\mathbb Z}
\newcommand{\loss }{\mathcal L}
\newcommand{\fun}{\operatorname{fun}}
\newcommand{\funnn}[1]{\fun([#1], [#1])}
\newcommand{\Loss}[1]{\mathcal L _\text{#1}}
% Lahko se zgodi, da je ukaz \C definiral že paket hyperref,
% zato dobite napako: Command \C already defined.
% V tem primeru namesto ukaza \newcommand uporabite \renewcommand
\newcommand{\C}{\mathbb C}
\newcommand{\Q}{\mathbb Q}

%%%%%%%%%%%%%%%%%%%%%%%%%%%%%%%%%%%%%%%%%%%%%%%%%%%%%%%%%%%%%%%%%%%%%%%%%%%%%%%
% ZAČETEK VSEBINE
%%%%%%%%%%%%%%%%%%%%%%%%%%%%%%%%%%%%%%%%%%%%%%%%%%%%%%%%%%%%%%%%%%%%%%%%%%%%%%%

\begin{document}

\section*{Uvod}
\addcontentsline{toc}{section}{Uvod}
Napišite kratek zgodovinski in matematični uvod.  Pojasnite
motivacijo za problem, kje
nastopa, kje vse je bil obravnavan. Na koncu opišite tudi
organizacijo dela -- kaj je v
katerem razdelku.

Motivacija: iskanje diskretnih struktur v matematiki (upodobitev,
delovanj, izomorfizmov grafov). Razvoj in uspeh gradientnih metod.
Iskanje morfizmov z gradientnimi metodami.

Zgodovina: Izračunljivost in časovna zahtevnost determinističnih
metod. Uporaba ml do zdaj.

Originalni rezultati: Naši izsledki. Isti setup za vse tri stvari.
Implementacija.

Struktura: \TODO{dopiši na koncu}
\section*{Oznake}
\addcontentsline{toc}{section}{Oznake}

\label{section:oznake}
\begin{tabularx}{\linewidth}[h!]{l p{0.9\linewidth}}
  $[n]$ & Množica števil $\{1, 2, \dotsc, n\}$. \\
  $\operatorname{flatten}$ &  Sploščitev matrike
  $\operatorname{flatten} \colon \R^{n \times n} \to \R^{n^2}$.
  % \newline
  Za $A=[a_{i,j}] \in \R^{n \times n}$ je $\operatorname{flatten}(A)
  _i = a_{\lfloor\frac{i}{n} \rfloor, i \bmod n}$.
  % $\operatorname{flatten} \colon \R^{n \times n} \to \R^{n^2}$
  \\
  $\operatorname{softmax}$ & Funkcija $\operatorname{softmax} \colon
  \R^n \to \R^n$. Glej definicijo \ref{def:softmax}.\\
  % $\operatorname{softmax}(x)_i = \frac{e^{x_i}}{\sum_{j=1}^n
  % e^{x_j}}$. Za definicijo na matrikah glej definicijo \ref{def:softmax}. \\
  %
  % $\phi$ & Parametri modela. \\
  %
\end{tabularx}
%
%
\clearpage
\section{Optimizacija z gradientnim spustom}
Na različne probleme v matematiki lahko gledamo kot na iskanje
minimuma $\hat \phi = \operatorname{argmin}( \mathcal L(\phi) \mid
\phi )$ funckionala $\loss$. Če je domena funkcionala evklidski
prostor, $\mathcal{L}\colon \R^n \to \R$ pa gladka, lahko minimum
iščemo z gradientnimi metodami optimizacije.

V tem poglavju definiramo numerične gradientne optimizacijske metode.
Definicije sledijo \cite{prince2023understandingdeeplearning}.
\subsection{Gradientni spust}
% Definicije gradientnih metod v tem delu sledijo
\begin{definicija}
  Naj bo $\mathcal{L} \colon \R^n  \to \R$ odvedljiva funkcija in
  $\phi_0 \in \R^n$ poljubna začetna vrednost parametrov.
  \emph{Gradientni spust}\index{gradientni spust} je iterativna
  optimizacijska metoda, definirana s pravilom
  \begin{equation}
    \label{eq:gradientni spust}
    \phi_{i+1} = \phi_i - \eta \nabla \mathcal{L}(\phi_i)
  \end{equation}
  Parametru
  $\eta > 0$ rečemo   hitrost učenja (\emph{learning rate})
  gradientnega spusta in vpliva na velikost spremembe parametrov.
\end{definicija}

\begin{primer}
  Na sliki~\ref{fig:primer gradientnega spusta} sta prikazana grafa
  rezultatov gradientnega spusta na funkciji $f(x) = x e^{\sqrt{x} -
  2x\sin(x)}$ z začetnim parametrom $x_0 = 1$ in različnima
  hitrostima učenja. Spremembe parametrov sledijo smeri padanja
  krivulje in so večje, če krivulja pada hitreje.
  \begin{figure}[ht]
    \centering
    \includegraphics[width=0.9\textwidth]{images/im:gradient_descent_example.pdf}
    \caption[Primer delovanja gradientnega spusta.]{$100$ korakov
      gradientnega spusta na funkciji $f(x) = x e^{\sqrt{x} -
      2x\sin(x)}$ z začetnim parametrom $x_0=1$ in hitrostjo učenja
      $\eta$. V obeh primerih metoda konvergira do lokalnega minimuma
    funkcije.}
    \label{fig:primer gradientnega spusta}
  \end{figure}
\end{primer}

\subsubsection{Gradientni tok}
\index{gradientni tok}
Naj bosta $\mathcal{L}$ in $\phi$ kot zgoraj. \emph{Gradientni tok}
je rešitev diferencialne enačbe
\begin{equation}
  \label{eq:gradient flow}
  \frac{d\phi}{d t} = - \nabla \mathcal{L}(\phi), \quad \phi(0)=\phi_0.
\end{equation}

Gradientni spust je ekvivalenten numeričnem reševanju diferencialne
enačbe \emph{gradientnega toka} z Eulerjevo metodo (glej
\cite{dherin2025learningsolvingdifferentialequations}).

\begin{opomba}
  V globokem učenju (\TODO{referenca na deep learning}) se za
  optimizacijo uporabljajo
  metode, ki temeljijo na gradientnem spustu, a nimajo nujne povezave
  z gradientnim
  tokom. Med drugim lahko z uporabo šuma v izračunu gradienta
  dobimo metodo, ki z visoko verjetnostjo pobegne iz sedel
  \cite{jin2017escapesaddlepointsefficiently}. Te lastnosti
  nima nobena rešitev enačb gradientnega toka.

  Alternativno bi se lahko optimizacije lotili z boljšimi numeričnimi
  metodami za reševanje gradientnega toka, npr Runge-Kutta metodami višjega reda
  \cite{dherin2025learningsolvingdifferentialequations}.
\end{opomba}

\subsection{Lastnosti gradientnega spusta}
V primeru~\ref{fig:primer gradientnega spusta} metoda konvergira do
lokalnega minimuma funkcije.
V splošnem je konvergenca odvisna od izbire  $\eta$ in začetnih
vrednosti parametrov $\phi_0$.

Rešitev $\phi(t)$ gradientnega toka $  \frac{d\phi}{d t} = - \nabla
\mathcal{L}(\phi), \quad \phi(0)=\phi_0$ vedno konvergira do
kritične točke $\loss$ (\cite[stran 203]{Hirsch2012dynamical-systems} ).
Limita $\lim \limits_{t \to \infty} \phi(t)$
je odvisna od izbire začetnih parametrov $\phi_0$.
% Z uporabo nevronskih mrež in večanjem števila parametrov lahko
%
% To ne velja v splošnem, kjer je konvergenca metode odvisna od
% hitrosti učenja $\eta$. Če je $\eta$ prevelik, je lahko sprememba
% parametrov prevelika, posledično pa se vrednost funckionala lahko poveča.
% V \cite{prince2023understandingdeeplearning} avtor to lastnost izkorišča
\subsection{Moment}
Kljub gladkosti funckije $\loss$ so lahko spremembe parametrov pri
uporabi gradientnega spusta kaotične. lahko jih delno omejimo z
uporabo \emph{momenta} (\cite[stran 86]{prince2023understandingdeeplearning})
\begin{align*}
  m_{i+1} &= \beta \cdot m_t + (1-\beta) \nabla \loss (\phi_i) \\
  \phi_{i+1} &= \phi_i -  \alpha m_{i+1},
\end{align*}
kjer sta $\beta \in [0,1), \alpha > 0$ parametra metode. Sprememba
parametra $\phi_{i+1} - \phi_i$ je s tem manj kaotična in bolj gladka.
% \TODO{To sem mislil dodati kot motivacijo za Adama. A jo sploh potrebujemo?}
\subsection{Adam}
\label{adam}
\emph{Adam} (\cite[stran 88]{prince2023understandingdeeplearning}) je
iteracijski optimizacijski algoritem, podan s pravili
\begin{align*}
  m_{i+1} &= \beta \cdot m_i + (1-\beta) \nabla \loss (\phi_i) \\
  \overline{m}_{i+1} &= \frac{m_{i+1}}{1 - \beta ^{i+1}}\\
  v_{i+1} &= \gamma \cdot v_t + (1-\gamma) (\nabla \loss (\phi_i))^2 \\
  \overline{v}_{i+1} &= \frac{v_{i+1}}{1 - \delta ^{i+1}}\\
  \phi_{i+1} &= \phi_i -  \alpha
  \frac{\overline{m}_{i+1}}{\overline{v}_{i+1} + \epsilon},
\end{align*}
kjer so $\alpha > 0, \beta \in [0,1), \delta \in [0,1 )$ parametri modela.

Zaradi dobrih empiričnih rezultatov se v praksi v strojen učenju
namesto surovega gradientnega spusta za optimizacijo uporablja
algoritem Adam.
%%%%%%%%%%%%%%%%%%%%%%%%%%%%%%%%%%%%%%%%%%%%%%%%%%%%%%%%%%%%%%%%%%%%%%%%%%%%%%%
%
\clearpage
\section{Iskanje upodobitev}
Obstaja očitna\footnote{Za $G=\{g_1, g_2, \dotsc, g_m\}$ lahko
  enačimo $f = (\operatorname{flatten}(f(g_1)),
    \operatorname{flatten}(f(g_2), \dotsc,
    \operatorname{flatten}(f(g_{m}) )$.} $1:1$ korespondenca med
    preslikavami končnih grup v evklidske prostore  $f\colon G \to
    \R^{n \times n} $ in prostorom $K = \R^{n^2 |G|}$.

    Definiramo lahko gladko funkcijo $\loss \colon K \to [0,
    \infty)$, za katero je $\loss (x) = 0$ natanko tedaj, ko $x$
    predstavlja upodobitev končne grupe.
    Iskanje upodobitev lahko prevedemo na minimiziranje preslikave $\loss(f)$.

    V poglavju \ref{subsection:upodobitve končnih grup} predstavimo
    potrebno teoretično ozadje. V \ref{subsection: Iskanje upodobitev
    z gradientnim spustom} natančno definiramo metodo za iskanje
    upodobitev. V poglavjih
    \ref{subsection : ciklicčne grupe} in \ref{subsection: diedrske
    grupe} študiramo dinamiko diferencialnih enačb gradientnega toka
    za ciklične in diedrske grupe.
    \subsection{Upodobitve končnih grup}
    \label{subsection:upodobitve končnih grup}
    Jezik teorije upodobitev je povzet po \cite{jezernik2025upodobitve}.
    \begin{definicija}
      \index{upodobitev}
      Naj bo $G$ grupa in $V$ vektorski prostor nad poljem
      $\mathbb{F}$. \emph{Upodobitev grupe $G$} je homomorfizem $
      \rho \colon G \to \operatorname{GL(V)}$ med grupo $G$ in
      endomorfizmi prostora $V$.
    \end{definicija}
    Vsaka upodobitev $\rho \colon G \to \operatorname{GL}(V)$ naravno
    definira linearno delovanje $v ^ g = \rho(g)(v)$ grupe nad $V$.
    Ekvivalentno vsako linearno delovanje  definira upodobitev
    $\rho(g) = (v \mapsto v^g)$.
    \begin{definicija}
      Upodobitvi $\rho \colon G \to \operatorname{GL}(V)$ in
      $\sigma \colon G \to \operatorname{GL}(W)$ sta
      \emph{ekvivalentni}, če obstaja izomorfizem vektorskih
      prostorov $\Phi \colon  V \to W$, da za vsak $g \in G$ in vsak
      $v \in V$ velja
      $$
      \Phi(\rho(g)(v)) = \sigma(g)(\Phi(v)).
      $$
      Preslikavi $\Phi$ rečemo \emph{spletična} med $V$ in $W$.
    \end{definicija}
    \begin{definicija}
      Naj bo $\rho: G \to \operatorname{GL}(V)$ upodobitev in $W < V$
      podprostor prostora $V$, invarianten za inducirano delovanje.
      Potem je upodobitev
      $
      \rho_W \colon G \to \operatorname{GL(W)}
      $, dana s predpisom
      $\rho_W(g) = \rho(g)|_{W}$ \emph{podupodobitev} upodobitve $\rho$.
    \end{definicija}
    \begin{definicija}
      \index{nerazcepna upodobitev}
      Če sta edini podupodobitvi upodobitve $\rho \colon G \to
      \operatorname{GL}(V)$ $g \mapsto \operatorname{id}$ in $\rho$,
      je $\rho$ \emph{nerazcpena upodobitev}.
      Množico nerazcepnih upodobitev grupe $G$ označimo z
      $\operatorname{Irr}(G)$.
    \end{definicija}
    \begin{definicija}
      Upodobitev $\rho \colon G \to \operatorname{GL}(V)$ je
      \emph{polenostavna}, če jo lahko izrazimo kot vsoto nerazcpenih
      upodobitev $\rho = \bigoplus _{i \in I} \rho_i$.
    \end{definicija}
    Če se omejimo le na končne grupe in dovolj lepa polja, so vse
    upodobitve sestavljene iz nerazcepnih.
    \begin{izrek}%{Polenostavnost upodobitev}
      Naj bo $G$ končna grupa in $\mathbb{F}$
      polje. Vse upodobitve $G$ nad poljem $F$ so polenostavne če in
      samo če $\operatorname{char}F \nmid |G|$.
    \end{izrek}
    \begin{izrek}
      Naj bo $G$ končna grupa in $\rho \colon G \to
      \operatorname{GL}(V)$ končno-razsežna upodobitev nad
      algebraično zaprtim poljem s karakteristiko $0$. Potem velja
      \begin{equation}
        \label{eq:nerazcepna iff norma=1}
        \rho \in \operatorname{Irr}(G) \iff |\chi_\rho| = \frac{1}{|G|}
        \sum\limits_{g \in G} \operatorname{tr}(\rho(g))
        \operatorname{tr}(\rho(g^{-1}) = 1.
        \end{equation}
      \end{izrek}
      Za študij upodobitev končnih grup nad ugodnimi polji je torej
      dovolj poznati le končno\footnote{Glej \cite[poglavje
          \emph{Dekompozicija regularne
      upodobitve}]{jezernik2025upodobitve}.} množico nerazcepnih
      upodobitev $\operatorname{Irr}(G) = \{ \rho \colon G \to
      \operatorname{GL}(V) \mid |\chi_\rho| = 1\}$.
      \subsection{Iskanje upodobitev z gradientnim spustom}
      \label{subsection: Iskanje upodobitev z gradientnim spustom}
      Naj bo $G=<S|R>$ končna grupa nad množico generatorjev $S$,
      definirana z relacijami $R$. Na $S$ definiramo poljubno preslikavo
      \begin{align}
        \label{eq: repr model na S}
        \hat \rho \colon S &\to \C^{n \times n}    \\
        s &\mapsto
        \begin{bmatrix}
          s_{1,1} & \cdots & s_{1,n} \\
          \vdots & & \vdots \\
          s_{n,1} & \cdots & s_{n,n}
        \end{bmatrix},
      \end{align}
      ki vsak generator grupe preslika v poljubno matriko.
      Preslikavo $\hat \rho$ implicitno razširimo na celo grupo. Za
      generator $s \in S$ nastavimo $\hat \rho(s^{-1}) =
      \hat\rho(s)^{-1}$, za poljuben produkt elementov $s_1, s_2,
      \dotsc, s_n \in S \cup S^{-1}$
      pa definiramo
      \begin{equation}
        \label{eq: implicitna definicija upodovbitve}
        \hat \rho(s_1 s_2 \dotsm s_m) = \hat\rho(s_1)\hat\rho(s_2)
        \dotsm \hat\rho(s_m).
      \end{equation}
      Preslikava $\hat \rho$ ni nujno upodobitev, služi pa nam lahko
      kot model upodobitve, odvisen od parametrov $\phi = \{s_{i,j}
      \mid s \in S, 1 \leq i, j \leq n\}$.
      Parametre matrik $\hat\rho(s)$ bomo postopoma spreminjali s
      gradientnim spustom, da bo model vse bližje upodobitvi.

      \subsubsection{Funkcija izgube nad relacijami}
      \label{funckija izgube nad relacijami}
      Preslikava $\hat \rho$ je homomorfizem natanko tedaj, ko za
      vsako relacijo $r \in R$ velja
      $
      \hat \rho(r) = I
      $. Manjše, kot so norme $||\hat \rho(r) - I ||_F$, bolj je
      model $\hat\rho$ podoben homomorfizmu.
      \begin{definicija}
        \emph{Funkcija izgube nad relacijami} je definirana kot
        \begin{equation}
          \label{eq:funkcija izgube nad relacijami            }
          \Loss{rel}(\hat\rho) = \frac{1}{|R|} \sum \limits_{r \in R}
          ||\hat \rho(r) - I  ||_F^2.
        \end{equation}
      \end{definicija}
      Očitno je model $\hat \rho$ upodobitev natanko tedaj, ko je $
      \Loss{rel}(\hat\rho) =0$.
      \begin{primer}
        Poglejmo si model $\hat \rho \colon S_3 \to \R$ eno
        dimenzijske realne upodobitve permutacijske grupe na treh
        točkah za prezentacijo $S_3 = <(12), (13)\mid
        (12)^2=(13)^2=(13)^{(12)}(12)^{(13)}=1>$. Model je definiran
        z vrednostmima
        $\hat\rho((12)) = x$ in $\hat\rho((13)) = y$.

        Edini enodimenzionalni upodobitvi $S_3$ sta identiteta
        $\operatorname{id}$ in predznak permutacije
        $\operatorname{sign}$. Na sliki so  \ref{fig:trajektorije S3
        to R} trajektorije gradientnega spusta pri optimizaciji
        $\Loss{rel}$ za različne začetne parametre $(x_0, y_0)$.
        Opazimo, nekatere trajektorije konvergirajo  do
        $\operatorname{id}$ nekatere do$\operatorname{sign}$,
        nekatere do modela, ki ni upodobitev. V tem primeru je
        gradientni spust našel lokalni ekstrem funkcije izgube, ki ni globalni.
        \begin{figure}[h]
          \centering
          \includegraphics[width=0.6\textwidth]{images/trajektorije_S3_to_R.pdf}
          % \caption[caption za v kazalo]{Dolg caption pod sliko}
          \caption[Trajektorje učenja modela $S_3 \to \R$.]{Različne
            trajektorije učenja modela $\hat\rho \colon S_3 \to \R$.
            Začetni parametri vplivajo na uspeh optimizacije in končno
          upodobitev. }
          \label{fig:trajektorije S3 to R}
        \end{figure}
      \end{primer}
      \begin{opomba}
        Model bi lahko definirali na celi grupi kot $\hat\rho(g)=
        \begin{bmatrix}
          g_{i,j}
        \end{bmatrix}_{i,j=1,\dotsc, n}$, za parametre pa vzeli
        elemente matrik $\hat\rho(g)$. V tem primeru je model
        homomomorfizem, če $\forall g, h \in G.
        \hat\rho(gh)=\hat\rho(g) \hat\rho(h)$. Za funkcijo izgube lahko vzamemo
        $$
        \Loss{homo} = \frac{1}{|G|^2}\sum_{i=1}^n \sum_{j=1}^n ||
        \hat\rho(gh)-\hat\rho(g) \hat\rho(h)||_F^2.
        $$
        in iskanje upodobitev spet prevedemo na optimizacijski problem.

        Ta pristop ni praktičen, saj privede do velikega števila
        parametrov $N=|G|^n$. Model upodobitve permutacijske grupe
        $S_n$ bi tako imel $(n!)^2$ parametrov, kar je neugodno za
        numerične simulacije. Če pa sledimo pristopu s prezentacijami
        in vzamemo minimalno množico generatorjev (recimo
        tranzpozicija in $n$-cikel), operiramo le s štirimi
        kompleksnimi parametri, neodvisno od velikosti grupe.
      \end{opomba}
      \subsubsection{Funkcija izgube za nerazcepnost}
      Iščemo lahko le nerazcepne upodobitve. Vemo že (izrek
      \ref{eq:nerazcepna iff norma=1}), da je upodobitev $\rho$
      nerazcepna, če je norma njenega \emph{karakterja} $\chi_\rho$ enaka 1.

      \begin{definicija}
        Definiramo funkcijo izgube za nerazcepnost
        \begin{equation}
          \label{eq:irr loss}
          \Loss{irr}(\rho) = (|\chi_\rho| - 1)^2.
        \end{equation}
      \end{definicija}
      Za poljuben model $\hat \rho$ lahko z gradientnim pustom
      minimaliziramo funckijo izgube $\Loss{rel} + \Loss{irr}$ in
      iščemo \emph{nerazcepne} upodobitve.
      \subsubsection{Funkcija izgube za unitarnost}
      Dodatno lahko zahtevamo, da naš model slika v unitarne matrike.
      Velja namreč\footnote{Glej \cite[trditev
      3.34]{jezernik2025upodobitve}.}, da je vsaka upodobitev
      ekvivalentna upodobitvi v unitarne matrike.
      S tem zmanjšamo število različnih upodobitev, h katerimi lahko
      konvergiramo.
      \begin{definicija}
        Funkcija izgube za unitarnost je
        \begin{equation}
          \label{eq: unitary loss}
          \Loss{unitary} = \frac{1}{|S|} \sum \limits_{s \in S}
          ||\hat\rho(s) \hat\rho(s)^H - I||.
        \end{equation}
      \end{definicija}
      Če ne piše drugače, od tu naprej vedno privzamemo, da
      minimiziramo funkcijo izgube $\loss = \Loss{rel} + \Loss{irr} +
      \Loss{unitary}$. Globalni minimumi te funkcije so natanko
      unitarne nerazcepne upodobitve.
      \subsubsection{Gradientni tok}
      V delu opazujemo dinamiko sistema $\frac{d\phi}{dt} = - \nabla
      \loss $. Zanimajo nas limite trajektorij in meje med območji
      začetnih parametrov z različnimi limitami.

      Diferencialne enačbe gradientnih tokov imajo nekatere lepe
      lastnosti. Limite različnih trajektorij (\emph{kritične točke
      sistema}) so natanko ničle $\nabla\loss(\phi) = 0$ gradienta
      (\cite[poglavje 9.3]{Hirsch2012dynamical-systems}).
      \begin{opomba}
        Enačbe gradientnih tokov nimajo vedno enostavnih  rešitev v
        zaprti obliki
        \cite{park2024absenceclosedformdescriptionsgradient}. Običajno se jih
        računa numerično.

        V delu enostavne\footnote{Za uporabo Runge-Kutte moramo
          poznati gradient $\nabla \loss$. Eksplicitni izračun $\nabla
          \loss $ z naraščajočim številom parametrov postane težaven in
          nepraktičen. V tem primeru se v delu numerične integracije
        lotimo z avtogradom, ki sam izračuna gradient.}  gradientne
        tokove (ciklične in diedrske grupe) integriramo s Runge-Kutto
        (s pythonovo
        metodo  \verb|scipy.integrate.solve_ivp.|), enačbe zapletenih
        gradientnih tokov rešujemo z avtograd implementacijo Adama
        (pythonova knjižnica \verb|torch|).
      \end{opomba}
      %%%%%%%%%%%%
      \subsection{Ciklične grupe}
      \label{subsection : ciklicčne grupe}
      Oglejmo si naš pristop iskanja nerazcepnih upodobitev na
      cikličnih grupah $C_n = <z | z^n=1>$. Edine nerazcepne
      upodobitve ciklične grupe so
      $$
      \rho_k(z) = e^\frac{2 \pi k}{n},
      $$
      kjer je $k \in \{0, 1, \dotsc, n\}$. Pri iskanju teh upodobitev
      z gradientnim spustom se bomo omejili na enodimenzionalne modele.

      Model $\hat\rho \colon G \to \C^*$ je določen z vrednostjo
      $\hat\rho(z) = x + iy$ in je odvisen od dveh realnih parametrov $x, y$.

      Funkcije izgube so

      \begin{align}
        \label{eq:loss function Cn}
        \Loss{rel} &= |(x + iy)^n -1|^2\\
        \Loss{irr} &= (\frac{1}{n} \sum_{i=0}^{n-1} |(x + iy)^i|^2 -1)^2 \\
        \Loss{unitary} &= | x^2 + y^2 -1|^2.
      \end{align}
      Na grafu \ref{fig:ciklicna-trajektorije} so trajektorije
      rešitev diferencialne enačbe $\frac{d(x,y)}{dt} = -\nabla
      \loss$ za različne začetne parametre, izračunane s pomočjo
      gradientnega spusta. Opazimo, da vse rešitve konvergirajo do
      nerazcepnih upodobitev. Ta rezultat teoretično podpremo v v
      izreku \ref{izrek:kritične točke ciklične}.
      \begin{figure}[ht]
        \centering
        \includegraphics[width=0.7\linewidth]{images/C3 trajektorije.pdf}
        \caption{Rešitve $\frac{d(x,y)}{dt} = -\nabla \loss$ pri
        različnih začetnih parametrih $(x_0, y_0)$ za $n=3$.}
        \label{fig:ciklicna-trajektorije}
      \end{figure}
      \subsection{Redukcija funkcij izgube}
      Nerazcepne upodobitve $\rho_k$ so edine enodimenzionalne
      upodobitve cikličnih grup. Za poljubno upodobitev $\rho \colon
      C_n \to \C^*$ mora namreč veljati $1 = \rho(z^n) = \rho(z)^n$,
      iz česar sledi, da je $\rho = \rho_k$ za nek $k$.
      Zato lahko pri iskanju nerazcepnih upodobitev z gradientnim
      spustom izpustimo $\Loss{irr}$.

      Ker so vse upodobitve $\rho_k$ unitarne, lahko iz funkcije
      izgube izpustimo tudi $\Loss{unitary}$ in za funkcijo izgube
      $\loss$ vzamemo le $\loss=\Loss{rel}$.

      V splošnem lahko pri optimizaciji funkcije izgube pričakujemo
      lokalne minimume, v katerih je vrednost $\loss$ neničelna.
      Edine privlačne točke enačbe $\frac{d(x,y)}{dt} = -\nabla ||(x
      + iy)³n -1||^2$ pa so koreni enote $z^n=1$, torej natanko
      unitarne nerazcepne upodobitve.
      \begin{izrek}
        \label{izrek:kritične točke ciklične}
        Edine kritične točke preslikave $(x,y) \mapsto ||(x + iy)^n
        -1||^2$ so $e^\frac{2 \pi i k}{n}$  za $k \in \{0,1, \dotsc,
        n-1\}$ in $(0,0)$.
        % Točka $(0,0)$ je odbojna točka, ostale pa privlačne.
      \end{izrek}
      \begin{dokaz}
        Lokalni ekstremi funkcije $\Loss{rel}$ so natanko v ničlah
        njenega gradienta $\nabla \Loss{rel} =
        (\frac{d\Loss{rel}}{dx}, \frac{d\Loss{rel}}{dy})$.

        Uvedemo $z = x + iy$.
        %
        \footnote{Imaginarno število $i$ v tem primeru predstavlja index.
          Zanima nas gradient $\nabla \loss \in \R^2$. Za lažje
          računanje $\R^2$ identificiramo z $\C$.
          Za $u   \in \R$  $iu$ predstavlja točko $(0, u) \in \R^2$.
        Zato pri izpeljavi $\frac{d\loss}{dy}$ izpustimo $i$.}
        %
        Velja $\loss = \Loss{rel} = (z^n-1)(\overline{z}^n-1)$,
        $$\frac{d\Loss{rel}}{dx} = nz^{n-1}(\overline{z}^n -1) + (z^n
        -1) n \overline{z}^{n-1} = 2 \operatorname{Re}(n
        z^{n-1}(\overline{z}^n -1))
        $$
        in
        $$\frac{d\Loss{rel}}{dy} =
        2\operatorname{Im}(n z^{n-1}(\overline{z}^n -1))
        $$.
        Če velja $\nabla \loss = 0$, je $z^{n-1}(\overline{z} ^n -1)
        = 0$. Take vrednosti $z$ so natanko $x=y=0$ in $x + iy =
        e^\frac{2 \pi i k}{n}$ za $k \in \N$.
        % \TODO{$0$ odbojna. (hessian >0?)}
      \end{dokaz}

      \subsection{Diedrske grupe}
      \label{subsection: diedrske grupe}
      Opazujmo diedrsko grupo $D_n = <r, s|r^n=s^2 = (rs)^2 = 1>$.
      Enorazsežne upodobitve diedrske grupe so
      \begin{align*}
        \chi_{\varepsilon, \delta} \colon \quad
        s &\mapsto \varepsilon, \\
        r &\mapsto \delta,
      \end{align*}
      kjer je  \( \varepsilon \in \{-1, 1\} \), in $\delta \in \{1,
      -1^{n + 1}\}$.
      Dvorazsežne nerazcepne upodobitve so
      \begin{align*}
        \rho_k \colon  s &\mapsto
        \begin{bmatrix}
          0 & 1 \\
          1 & 0
        \end{bmatrix}\\
        r &\mapsto
        \begin{bmatrix}
          \cos\left(\frac{2\pi k}{n}\right) & -\sin\left(\frac{2\pi
          k}{n}\right) \\
          \sin\left(\frac{2\pi k}{n}\right) & \cos\left(\frac{2\pi k}{n}\right),
        \end{bmatrix}
      \end{align*}
      kjer je $0 \leq k < \frac{n}{2} $. Vse nerazcepne upodobitve so
      realne, zato jih bomo iskali z realnimi modeli.
      \subsubsection{Enodimenzionalne upodobitve}
      Model enorazsežne upodobitve diedrske grupe
      je določen s slikama generatorjev\footnote{Izrabljamo notacijo
        in z $r$ označimo abstraktni element grupe $D_n$ in realno
      število $r \in \R$, ki predstavlja sliko generatorja $r$.}
      %
      $\hat \rho(s) = s \in \R, \hat \rho(r)=r \in \R$. Funkcije izgube so enake
      \begin{align*}
        \Loss{rel} &= \frac{1}{3} \left( |s^2 -1|^2 + |r^n -1|^2  +
        |(rs)^2 -1|^2  \right )\\
        \Loss{unit} &=  \frac{1}{2} \left ( |r^2 -1|^2 + |s ^2 -1|^2
        \right)  \\
        \Loss{irr} &= \left ( \frac{1}{2n}\sum_{i=0}^n(|r^i|^2 +
        |r^is|^2)    - 1\right )^2.
      \end{align*}
      \paragraph{Numerične rešitve}
      Numerično rešujemo enačbo gradientnega toka
      $$\frac{d(r,s)}{dt} = -\nabla(\Loss{rel} + \Loss{unit} +
      \Loss{irr}).$$ Za reševanje NDE uporabimo \verb|scipy.solve_ivp()|.
      % Implementacija simulacije je v \TODO{dodatek?}.

      Na sliki \ref{fig:Dn-trajektorije-demo} so različne rešitve
      enačb gradientnega toka za $n\in \{1,2,7,8\}$. Modre v limiti
      dosežejo ničelno funkcijo izgube. Za sod $n$ vse narisane
      trajektorije konvergirajo do nerazcepne upodobitve, za lih $n$
      pa gradientni spust iz $[-1,0]\times[-1,1]$ nikoli ne doseže
      ničelne napake. Enake lastnosti opazimo na širši množici
      diedrskih grup - glej \TODO{grafi za ostalo}
      %%%%%%%%%%%%%
      \begin{figure}[h!]
        \centering
        \begin{minipage}{0.49\textwidth}
          \centering
          \includegraphics[width=\linewidth]{images/dihedral/Dn_1dim_n1.pdf}
          \caption*{$n=1$}
        \end{minipage}
        \centering
        \begin{minipage}{0.49\textwidth}
          \centering
          \includegraphics[width=\linewidth]{images/dihedral/Dn_1dim_n2.pdf}
          \caption*{$n=2$}
        \end{minipage}
        \vspace{0.5em}
        \centering
        \begin{minipage}{0.49\textwidth}
          \centering
          \includegraphics[width=\linewidth]{images/dihedral/Dn_1dim_n7.pdf}
          \caption*{$n=7$}
        \end{minipage}
        \centering
        \begin{minipage}{0.49\textwidth}
          \centering
          \includegraphics[width=\linewidth]{images/dihedral/Dn_1dim_n8.pdf}
          \caption*{$n=8$}
        \end{minipage}
        \caption[Rešitve enačb gradientnega toka za različne $n$ in
          različne začetne parametre. Črne pike predstavljajo začetne
          vrednosti rešitev. Modre rešitve kovergirajo k nerazcepni
          upodobitvi, oranžne pa v lokalni minimum $\loss$, ki ni globalni.
        Več grafov je v \TODO{Dodatek}.]{Rešitve enačb gradientnega
          toka za različne $n$ in različne začetne
          parametre\footnote{Na vsakem grafu je 200 različnih
            rešitev. Začetni parametri so vzorčeni iz enakomerne
            porazdelitve nad $[-1,1]^2$. Za implementacijo glej
          \TODO{implementacija}}.
          Črne pike predstavljajo začetne vrednosti rešitev. Modre
          rešitve kovergirajo k nerazcepni upodobitvi, oranžne pa v
          lokalni minimum $\loss$, ki ni globalni.
        Več grafov je v \TODO{Dodatek}.}
        \label{fig:Dn-trajektorije-demo}
      \end{figure}
      %
      %
      %
      Opazimo, da vse numerične rešitve z začetnimi parametri $(r_0,
      s_0)$ konvergirajo k $\operatorname{sign}(r_0),
      \operatorname{sign}(s_0)$. Za vsak par začetnih parametrov
      $(r_0, s_0) \in \R^2$ enodimenzionalnega modela lahko študiramo
      hitrost konvergence h lokalnim ekstremom. Slika
      \ref{figure : limit points and speed for dihedral} prikazuje
      čas $t_{r_0, s_0}$, ob katerem je rešitev $(r(t), s(t))$ z
      začetnimi parametri $(r_0, s_0)$ prvič v $\epsilon = TODO$ pasu
      njene limite v neskončnosti.
      \begin{figure}[h!]
        \label{figure : limit points and speed for dihedral}
        \centering
        \includegraphics[width=0.8\linewidth]{images/dihedral/grid/plot_dihedral_grid_n4_-3_3_1000.pdf}
        \caption{Hitrosti konvergenc različnih začetnih parametrov.
          Barva točke $(r_0,s_0)$ določa limito - zelene točke
          konvergirajo k $(1,1)$, modre k $(1,-1)$, oranžne k $(-1,-1)$
          in rdeče k $(-1,1)$. Odtenek točke določa hitrost konvergence
          rešitve $(r(t), s(t))$. Svetlejša, kot je barva, manjši je
          prvi $t_1$, v katerem se $(r(t_1), s(t_1))$ razlikuje od
        limite za manj kot $\epsilon = 0.01$.}
        \label{fig:enter-label}
      \end{figure}
      \subsubsection{Dvodimenzionalne upodobitve}
      Model dvorazsežne upodobitve $D_n$ je določen z matrikama $\hat
      \rho(s) = S \in \R^{2\times 2}$ in $\hat \rho(r) = R \in \R^{n
      \times n}$. Funkcije izgube so enake
      \begin{align*}
        \Loss{rel} &= \frac{1}{3} \left( ||S^2 -1||_F^2 + ||R^n
        -1||_F^2  + ||(RS)^2 -1||_F^2  \right )\\
        \Loss{unit} &=  \frac{1}{2} \left ( ||R^2 -1||_F^2 + ||S ^2
        -1||_F^2  \right)  \\
        \Loss{irr} &= \left ( \frac{1}{2n}\sum_{i=0}^n(||R^i||_F^2 +
        ||R^iS||_F^2)    - 1\right )^2.
      \end{align*}
      \paragraph{Numerične rešitve}
      Slika \ref{fig:diedrska-2dim-trajektorije} prikazuje
      numerične rešitve za različne začetne parametre. Trajektorije
      rešitev $(R(t), S(t))$ vizualiziramo v $\R^2$ kot
      $(\operatorname{tr}(R(t), \operatorname{tr(S(t)})$.
          \begin{figure}[h!]
            \centering
            \begin{minipage}{0.49\textwidth}
              \centering
              \includegraphics[width=\linewidth]{images/dihedral/2d/prob_of_convergence_D_3.png}
              \caption*{$n=3$}
            \end{minipage}
            \begin{minipage}{0.49\textwidth}
              \centering
              \includegraphics[width=\linewidth]{images/dihedral/2d/prob_of_convergence_D_8.png}
              \caption*{$n=8$}
            \end{minipage}
            \caption[Rešitve za različne $n$. Modre krivulje
              konvergirajo do nerazcepnih upodobitev. Črne pike so
            začetni parametri]{Rešitve za različne $n$. Modre
              krivulje konvergirajo do nerazcepnih upodobitev. Črne
              pike so začetni parametri\footnotemark
              % \footnote{Začetni parametri $R_0, S_0 $ so oblike
              % $\begin{bmatrix}
              %     x & y \\
              %     z & \operatorname{tr} - x
              % \end{bmatrix}$, kjer so $x,y,z, \operatorname{tr} \sim U[-2,2]$}
            .}
            \label{fig:diedrska-2dim-trajektorije}
          \end{figure}
          \footnotetext{Začetni parametri $R_0, S_0 $ so oblike $
            \begin{bmatrix}
              x & y \\
              z & \operatorname{tr} - x
          \end{bmatrix}$, kjer so $x,y,z, \operatorname{tr} \sim U[-2,2]$}
          Nekateri začetni parametri vodijo do nerazcepne upodobitve,
          drugi ne. Računamo lahko verjetnost izbire začetnih
          parametrov, ki konvergirajo do ničel funkcije izgube.
          Verjetnosti v tabeli
          \ref{tab:verjetnost-konvergence-diedrska-2dim} so
          izračunane iz iz vzorcov v dodatku \TODO.
          \begin{table}[ht]
            \centering
            \begin{tabular}{c|c}
              Group &  $P\left( \loss (\lim \limits_{t \to
              \infty}\phi(t)) = 0  \right)$\\
              \hline
              $D_{3}$ & $0.274$\\
              $D_{4}$ & $0.158$\\
              $D_{5}$ & $0.306$\\
              $D_{6}$ & $0.21$\\
              $D_{7}$ & $0.311$\\
              $D_{8}$ & $0.235$\\
              $D_{9}$ & $0.33$\\
              $D_{10}$ & $0.241$\\
              $D_{11}$ & $0.327$\\
              $D_{12}$ & $0.248$\\
              $D_{13}$ & $0.319$
            \end{tabular}
            \caption{Verjetnost izbire začetnih parametrov, ki
              konvergirajo. Začetni parametri so vzorčeni enako kot v
            \ref{fig:diedrska-2dim-trajektorije}. Velikost vzorca je $10^4$.}
            \label{tab:verjetnost-konvergence-diedrska-2dim}
          \end{table}
          %%%%%%%%%%%%%%%%%%%%%%%%%
          %%%%%%%%%%%%%%%%%%%
          %  \subsection{Parametrizacija ortogonalne grupe}
          % \TODO{Študiramo $p \colon \R \to O_2$}.
          % out of scope for now
          %
          %
          %
          \clearpage%newpage pusti floatom, da so prej!
          \section{Iskanje delovanj}
          Na podoben način lahko iščemo delovanja grup. Naj bo $G$
          končna grupa in $[n] =    \{1,2,\dotsc, n\}$ končna
          množica. Iščemo homomorfizem $\rho \colon G \to S_n$.

          Vsako preslikava $f \in \funnn{n}$ se da predstaviti z vektorjem $
          \begin{bmatrix}
            f(1)\\
            \vdots \\
            f(n)
          \end{bmatrix} \in \R^n$. Lahko bi začeli s poljubno
          preslikavo $\hat \rho \colon S \to \R^n$ in sistematično
          spreminjali elemente vektorjem $\hat\rho(s)$, dokler ne
          pridemo do homomorfizma. To se da, a za optimizacijo
          parametrov ne moremo uporabiti gradientnega spusta, saj je
          prostor vektorjev $\{[f(i)]_{i=1, \dotsc n} | f \in
          \funnn{n}\}$ diskreten.

          Nad prostorom funkcij $\funnn{n}$ definiramo gladko družino
          porazdelitev. Generatorje grupe sprva slikamo v poljubne
          porazdelitve, nato pa spreminjamo porazdelitve, da je
          verjetnost za homomorfizem čim večja.\footnote{V strojnem
            učenju nad diskreten prostor izidov $D$ običajno
            definiramo gladko družino porazdelitev
            $\{\mathcal{P}_\phi(D)\}$ in minimiziramo  $-\log(P_\phi
          (\text{opaženi izidi}))$.}

          Model $\hat \rho$ delovanja $\rho \colon G \to S_n$ bo
          preslikava $\hat \rho \colon G \to \mathcal{P}(\funnn{n})$,
          definirana na generatorjih $s \in S$ s predpisom
          \begin{equation}
            \label{eq: model za delovanja}
            s \mapsto
            \begin{bmatrix}
              P(s(1) = 1) & P(s(1) = 2) & \dotsm & P(s(1) = n) \\
              P(s(2) = 1) & P(s(2) = 2) & \dotsm & P(s(2) = n) \\
              \vdots & \vdots & & \vdots \\
              P(s(n) = 1) & P(s(n) = 2) & \dotsm & P(s(n) = n) \\
            \end{bmatrix}
          \end{equation}
          in implicitno (glej enačbo  \eqref{eq: implicitna
          definicija upodovbitve}) drugje. Matrika $P_s =
          \hat\rho(s)$ definira slučajno preslikavo\footnote{Izraba
            notacije. S $s$ označimo tako generator  kot
            slučajno funkcijo nad $[n]$, ki je porazdeljena s
            porazdelitvijo $P(s = f) = \prod _{i=1}^n P(s(i) = f(i)$.
              Prav tako za poljuben element $g=s_1s_2\dotsm s_m \in G$
            z $g$ označimo tudi slučajno spremenljivko $s_1 s_2 \dotsm s_n$.}
            $s \in \funnn{n}$ s porazdelitvijo
            $$P(s = f) = \prod _{i=1}^n P(s(i) = f(i)$$
              za $f \in \funnn{n}$.

              Lahko je preveriti, da za  $g = s_1 s_2 \dotsm s_m\in G$ velja $$
              \hat \rho (g) = P_{s_1} P_{s_2} \dotsc P_{s_m} =
              \begin{bmatrix}
                P(g(1) = 1) & P(g(1) = 2) & \dotsm & P(g(1) = n) \\
                P(g(2) = 1) & P(g(2) = 2) & \dotsm & P(g(2) = n) \\
                \vdots & \vdots & & \vdots \\
                P(g(n) = 1) & P(g(n) = 2) & \dotsm & P(g(n) = n) \\
              \end{bmatrix}.
              $$
              Naš model elemente grupe slika v stohastične matrike,
              ki definirajo verjetnostne porazdelitve slučajnih
              funkcij nad $n$. Parametre modela bomo optimizirali
              tako, da bo homomorfizem najbolj verjeten.
              \subsection{Parametrizacija stohastičnih matrik}
              \label{section:parametrizacija stohastičnih matrik}
              Definicija modela $\hat  \rho$ zahteva, da so matrike
              $P_s$ stohastične\footnote{$P$ je stohastična, če je
                vsota elementov v vsaki vrstici enaka $1$, vsi elementi
              matrike pa so med $0$ in $1$.}. Iteracije gradientnega
              spusta stohastičnosti ne ohranjajo. Če hočemo za
              optimizacijo modela $\hat \rho$ uporabiti gradientni
              spust, moramo metodo prilagoditi tako, da novi
              parametri ostanejo v družini stohastičnih matrik.

              Družino stohastičnih matrik lahko parametriziramo s
              poljubno parametrizacijo $p \colon \R^{n \times n} \to
              \{P \in \R^{n \times n} \mid P \text{ je stohastična
              }\}$, stohastične matrike $P_s$ pa definiramo kot
              $p(\phi_s)$, kjer je $\phi = \{\phi_s | s\in S \}
              \subset \R^{n \times n}$ množica poljubnih matrik.
              \emph{Za parametre modela} vzamemo elemente matrik
              $\phi_s$. Na ta način bodo matrike $P_s = P_s(\phi_s)$
              vedno stohastične, ne glede na spremembo parametrov
              $-\eta \nabla_\phi \loss(P)$.

              Za parametrizacijo $p$ lahko vzamemo preslikavo
              $\operatorname{softmax}$, ki vsako vrstico spremeni v
              stohastični vektor.
              \begin{definicija}
                \label{def:softmax}
                Preslikava $\operatorname{softmax} \colon \R^n \to
                \R^n$ slika vektor $v = [v_1, v_2, \dotsc, v_n]^T$ v
                vektor z elementi $\operatorname{softmax}(v)_i =
                \frac{e^{v_i}}{\sum_{j = 1}^n e^{v_j}}$.

                Za matriko $A =
                \begin{bmatrix}
                  a_1^T \\
                  a_2^T \\
                  \vdots \\
                  a_n^T
                \end{bmatrix}$ definiramo \emph{softmax po vrsticah} kot
                $$\operatorname{softmax}(A) =
                \begin{bmatrix}
                  \operatorname{softmax}(a_1)^T \\
                  \operatorname{softmax}(a_2)^T \\
                  \vdots \\
                  \operatorname{softmax}(a_n)^T
                \end{bmatrix}.$$
              \end{definicija}
              Preslikava $\operatorname{softmax}$ očitno poljubne
              matrike preslika v stohastične. Z njo lahko za
              generator $s \in S$ matriko $P_s$ definiramo kot
              $$
              P_s = \operatorname{softmax}(\phi_s).
              $$
              Model za delovanja $\hat \rho = \hat \rho_\phi$
              definiramo v odvisnosti od parametrov $\phi = \{\phi_s
              \in \R^{n \times n } \mid s \in S\}$ kot
              $$
              \phi \mapsto ( s \mapsto \operatorname{softmax}(\phi_s))
              $$
              in spreminjamo elemente matrik $ \phi_s$, da model
              $\hat \rho_\phi$ skoraj zagotovo predstavlja delovanje.

              \subsection{Funkcija izgube nad relacijami}
              Naj bo $<S|R>$ prezentacija grupe $G$. Za vsako
              relacijo $r \in R$ bi radi $P(r = \operatorname{id})
              =1$ oziroma $0 = \log(P(r=\operatorname{id})) =
              \log(\prod_{i=1}^n P(r(i)=i)) =
              \operatorname{tr}(\log(P_r))$, kjer $\log(P_r)$
              računamo po elementih.
              \begin{definicija}
                \emph{Funkcijo izgube nad relacijami} za delovanja
                definiramo kot
                \begin{equation}
                  \label{eq:relation loss actions}
                  \Loss{rel} (\hat\rho) = - \frac{1}{|R|} \sum_{r \in
                  R} \operatorname{tr}(\log(P_r)).
                \end{equation}
              \end{definicija}
              \subsection{Pretvorba modela v delovanje}
              \TODO{Ko loss skovnergira do $\loss < \epsilon$, za
                delovanje vzamemo kar
                najbolj verjetno preslikavo iz porazdelitve. Naj bo
                permutacijska matrika preslikave generatorja $s$,
                ki je podan z matriko $P_s$, označena s $\tilde{P_s}$.

                Če je $\loss$ majhen, potem se za relacijo $r$ da dokazati?
                $$
                P_r = I + E_r,
                $$
                kjer je $(E_r)_{i,j} < \varepsilon$.

                Treba je dokazati, da je preslikava $s \mapsto
                \tilde{P_s}$ homomorfizem, torej da je
                za $r = s_1 s_2 \dotsm s_m$
                $$
                \tilde{P_{s_1}} \tilde{P_{s_2}} \dotsm \tilde{P_{s_m}} = I
                $$
                S tem imamo zagotovilo, da lahko za zelo majhen loss
                iz naše metode sploh dobimo delovanje.
              }
              %
              %
              \paragraph{Spomin:} Za grupo
              $G = <S|R>$  definiramo model 
              $s \mapsto P_s$  med generatorji in \textbf{stohastičnimi} matrikami.
              Za element grupe $g = s_1 s_2 \dotsm s_m \in G$ definiramo
              $P_g = P_{s_1} P_{s_2} \dotsm P_{s_m}$. In minimiziramo
              funkcijo izgube
              $$
              \sum_{r \in R} |P(r =\operatorname{id}) - 1|.
              $$ 
%
%
              \begin{lema}[Z minimaliziranjem $\loss$ je lahko $P_r$
                poljubno blizu $I$.]
                \label{lema:Pr_blizu_I}
                Za $\epsilon > 0$ obstaja $\delta>0$, da iz   $1 -
                P(r = \operatorname{id}) < \delta$ sledi $P_r = I +
                E_r$ za $||E_e||_F < \epsilon$
              \end{lema}
              \begin{dokaz}
                Ocenimo
                $$ 1 - (P_r)_{i,i} \leq  1 - (P_r)_{1,1}
                (P_r)_{2,2} \dotsm (P_r)_{n,n}
                = 1 - P(r=\operatorname{id})< \delta.$$
                %
                $P_r$ je stohastična, torej velja
                $\sum_{j=1}^n (P_r)_{i,j} = 1$. Iz tega sledi
                $\sum_{j \neq i}
                (P_r)_{i,j} < \delta$. Sledi
                $$
                P_{i, j} < \sum_{j \neq i} P_{i,j} < \delta,
                $$
                za $i \neq j$. Iz tega sledi $||E_r||_F < n
                \delta$.
              \end{dokaz}
              %%%%%%%%%%
              \begin{lema}[Za dovolj majhen $\epsilon$ je $P_s$ obrlnjiva]
                Obstaja $\epsilon_0$, da za vsak $0 < \epsilon <
                \epsilon_0$ in $P_r = I + E_r$ z $||E_r||_F^2 <
                \epsilon$ velja, da je $P_s$ obrnjiva za vsak
                generator $s \in S$.
              \end{lema}
              \begin{dokaz}
                Vemo, da obstaja $\varepsilon_0$, da je za
                $||E||_F < \epsilon$ matrika $I + E$ obrnljiva. (Vzameš zvezno preslikavo
                  $f(A) =  \det (I + A)$ in opaziš, da je
                  $0 \in f^{-1}(\frac{1}{2}, \frac{3}{2})$, iz česar
                  zaradi zveznosti sledi,
                  da $K_\varepsilon(0) \in  f^{-1}(\frac{1}{2},
                  \frac{3}{2})$ za dovolj majhen $\varepsilon > 0$.
                )

                Naj bo $r = s_1 s_2 \dotsm s_m \in R$. Za dovolj
                majhen $\epsilon$ je
                $P_r = I + E_r = P_{s_1}  P_{s_2} \dotsm  P_{s_n}$
                obrnljiva, torej so vse $ P_{s_i}$ obrnljive.
              \end{dokaz}
              %%%%%%%%%%%
              \begin{lema}[Če je permanent (stohastične) matrike
                blizu $1$, je matrika skoraj permutacijska]
                Naj bo $S$ stohastična matrika in $\epsilon>0$. Če je
                $\operatorname{Perm}(S) = 1 - \delta$ za dovolj
                majhen $\delta >0$, potem je $S = \overline{S} + E$
                in $||E||_F^2 < \epsilon$.
              \end{lema}
              \begin{dokaz}
                Naj bo $\mathcal{S} \subset \R^{n \times n}$ množica
                stohastičnih matrik. Očitno je kompaktna.

                Najprej dokažimo, da je $S = P + E$ za neko permutacijsko
                matriko $P$ in  $||E||_F < \epsilon.$.

                Opazujmo zvezno preslikavo $\operatorname{Perm}
                \colon \mathcal{S} \to \R$.
                Množica $\Pi  \subset \mathcal{S}$ permutacijskih
                matrik je kompakt v $\mathcal{S}$. Definiramo množico
                $$
                U = \{ A \in \mathcal{S} \mid \operatorname{d}(\Pi,
                A) > \epsilon  \}
                $$
                in opazimo, da je kompaktna. Označimo
                $$
                \max _{A \in U} \operatorname{Perm}(A) = 1 - \delta_0 < 1
                $$
                %
                - zadnja neenakost sledi, ker so maksimumi
                $\operatorname{Perm}$ natanko $Pi$.

                Naj bo stohastična matrika $S \in \mathcal{S}$ taka, da je
                $\operatorname{Perm}(S) > 1 - \delta_0$. Potem $S \notin U$,
                torej obstaja permutacijska matrika $P$, da je
                $$
                \operatorname{d}(S, P) = ||S - P || \leq \epsilon.
                $$
                Definiramo $E = S - P$. Sledi $||E||_F < \epsilon$.
                Očitno je $P = \overline{S} = \overline{P + E} $ za vsak
                $\epsilon < \frac{1}{2}$.
                %
              \end{dokaz}
              %
              %%%%%%%%%
              \begin{trditev}[Če sta $\Loss{rel}$ in $\Loss{bij}$
                majhna, je $s \mapsto \overline{P_s}$ homomomorfizem]
                Obstajata $\delta_1, \delta_2 > 0$, da iz
                %
                $\operatorname{Perm}(P_s) <  \delta_1$ za vsak
                generator $s \in S$ in
                $P(r \in R) < \delta_2$ za vsak $r \in R$ sledi, da je
                preslikava $f \colon s \mapsto \overline{P_s}$ homomorfizem.
              \end{trditev}
              \begin{dokaz}
                Izberemo $\epsilon >0$.
                Naj bo $r = s_1 s_2 \dotsm s_m \in R$ taka, da je
                $P_r = I + E_r$ in $||E_r||_F^2 < \epsilon$.
                Naj bodo $P_{s_i} = \overline{P_{s_i}} + E_{s_i}$ za
                $||E_{s_i}||_F^2 < \epsilon$.

                Dokažimo, da je $$\overline{P_{s_1}} \overline{P_{s_2}}
                \dotsm \overline{P_{s_m}} = I.$$

                Velja
                \begin{align*}
                  P_r &= P_{s_1} P_{s_2} \dotsm P_{s_m}
                  \\
                  &=
                  (\overline{P_{s_1}} + E_{s_1})
                  (\overline{P_{s_2}} + E_{s_2}) \dotsm
                  (\overline{P_{s_m}} + E_{s_m})
                  \\
                  &=
                  \overline{P_{s_1}} \overline{P_{s_2}} \dotsm
                  \overline{P_{s_m}} +
                  \sum_{Z \subsetneqq [m]} \prod_{i \in Z} \overline{P_{s_i}}
                  \prod_{j \notin Z} E_{s_j}
                  \\
                  &= I + E_r.
                \end{align*}
                %
                Ocenimo
                \begin{align*}
                  ||\prod_{i=1}^m \overline{P_{s_i}} - I ||
                  &= ||E_r - \sum_{Z \subsetneqq [n]} \prod_{i \in Z}
                  \overline{P_{s_i}}
                  \prod_{j \notin Z} E_{s_j} || \\
                  &\leq
                  ||E_r|| +
                  \sum_{Z \subsetneqq [m]} \prod_{i \in Z}
                  ||\overline{P_{s_i}}|| \prod_{j \notin Z} ||E_{s_j} || \\
                  &\leq \epsilon  +
                  \sum_{Z \subsetneqq [m]} \prod_{i \in Z}
                  \sqrt{m} \prod_{j \notin Z} \epsilon
                  \\
                  &=
                  \epsilon(1 + m^\frac{3}{2}(2^{m-1}-1))
                  \xrightarrow{\epsilon \to 0} 0.
                \end{align*}
                Ker sta $\prod_{i=1}^m \overline{P_{s_i}}$ in $I$ permutacijski
                matriki, je norma njune razlike vsaj $1$, če nista enaki.
                %
                %
                %
                %
                %
                %
              \end{dokaz}
              %%
              \subsection{Delovanje ciklične grupe}
              Študiramo model za delovanje ciklične grupe $C_n = <z |
              z^n=1>$ na končni množici $[m] = \{1,2,\dotsc, m\}$.
              Minimaliziramo
              $$
              \loss(\phi) = - \operatorname{tr}(\log
              \left(\operatorname{softmax}(Z^n) \right ) ),
              $$
              kjer je $Z \in \R^{m \times m}$.
              \paragraph{Rezultati}
              Na sliki \ref{fig:actions-cn} so predstavljene začetne
              in končne vredosti matrike $P(Z)$ pri iskanju delovanj
              ciklične grupe $C_n$. Podatki so bili priodobljeni z
              \TODO{Adam, settings, kaj vse}. Več rezultatov je v
              dodatku \TODO{}.
              \begin{figure}[h!]
                \centering
                \begin{minipage}{0.49\textwidth}
                  \centering
                  \includegraphics[width=\linewidth]{images/actions/C6_on_6_sample8_initial.pdf}
                  \caption*{$n=6, m=6$}
                \end{minipage}
                \begin{minipage}{0.49\textwidth}
                  \centering
                  \includegraphics[width=\linewidth]{images/actions/C6_on_6_sample8_final.pdf}
                  \caption*{Model konvergira k $(154)(236)$.}
                \end{minipage}
                \vspace{0.5em}
                %
                \centering
                \begin{minipage}{0.49\textwidth}
                  \centering
                  \includegraphics[width=\linewidth]{images/actions/C6_on_7_sample1_initial.pdf}
                  \caption*{$n=6, m=7$}
                \end{minipage}
                \begin{minipage}{0.49\textwidth}
                  \centering
                  \includegraphics[width=\linewidth]{images/actions/C6_on_7_sample1_final.pdf}
                  \caption*{Model konvergira k $(137)(456)$.}
                \end{minipage}
                %
                \vspace{0.5em}
                %
                \centering
                \begin{minipage}{0.49\textwidth}
                  \centering
                  \includegraphics[width=\linewidth]{images/actions/C6_on_7_sample2_initial.pdf}
                  \caption*{$n=6, m=7$}
                \end{minipage}
                \begin{minipage}{0.49\textwidth}
                  \centering
                  \includegraphics[width=\linewidth]{images/actions/C6_on_7_sample2_final.pdf}
                  \caption*{Model konvergira k izrojeni preslikavi.}
                \end{minipage}
                %
                \caption[Stohastične matrike na začetku in po \TODO
                korakih gradientnega spusta.]{Stohastične matrike na
                  začetku in po \TODO korakih gradientnega
                spusta.}
                \label{fig:actions-cn}
              \end{figure}
              \subsection{Delovanje diedrske grupe}
              Študiramo model za delovanje  grupe $D_n = <r,s |
              r^n=s^2=(rs)^2=1>$ na končni množici $[m] =
              \{1,2,\dotsc, m\}$. Minimaliziramo
              $$
              \loss(R, S) = - \operatorname{tr}(\log (P_r^n))
              - \operatorname{tr}(\log (P_s^n))
            - \operatorname{tr}(\log (P_rP_s)^2))
            ,
            $$
            kjer je $P_r = \operatorname{softmax(R)},$
            $P_s=\operatorname{softmax}(S)$ in    $R, S \in \R^{m \times m}$.
            \paragraph{Rezultati}
            \TODO{setup, popravi imena grafov}

            \subsection{Rekurzivni pristop}

            \TODO{DOdaj rekurzivni (strojnoučenjaški) pristop. Potestiraj ga}

            \clearpage
            \section{Iskanje izomorfizmov grafov}
            S podobnim pristopom kot v \ref{eq: repr model na S}
            lahko iščemo izomorfizme grafov. Iščemo model za
            bijektivno preslikavo $f \colon G_1 \to G_2$ med grafoma
            $G_1$ in $G_2$, za katero velja $f(i) \sim_2 f(j) \iff i \sim_1 j$.

            Za model lahko vzamemo slučajno preslikavo $f$, podano s
            stohastično matriko
            $$P_f =
            \begin{bmatrix}
              P(f(1) = 1) & P(f(1) = 2) & \dotsm & P(f(1) = n) \\
              P(f(2) = 1) & P(f(2) = 2) & \dotsm & P(f(2) = n) \\
              \vdots & \vdots & & \vdots \\
              P(f(n) = 1) & P(f(n) = 2) & \dotsm & P(f(n) = n) \\
            \end{bmatrix} $$
            in spreminjamo elemente matrike tako, da je $f $ vse bolj
            verjetno izomorfizem. Zato potrebujemo funkcijo izgube,
            ki je ničelna natanko v izomorfizmih - torej bijektivnih
            preslikavah, ki slikajo povezane vozlišča v povezane.
            \subsection{Funkcija izgube za bijektivnost}
            Naj bo $f$ slučajna preslikava iz $\funnn{n}$, definirana
            kot zgoraj. Radi bi, da je $f$ skoraj zagotovo
            bijektivna, torej da velja
            \begin{align*}
              1 &= P(f \in S_n) \\
              &= \sum_{\sigma  \in S_n} P(f =  \sigma) \\
              &= \sum_{\sigma  \in S_n} \prod_{i=1}^n (P_f)_{i, \sigma(i)} \\
              &= \operatorname{Perm(P_f)},
            \end{align*}
            kjer $\operatorname{Perm}(P_f)$ označuje \emph{permanent}
            \footnote{$\operatorname{Perm}(A) = \sum_{\sigma  \in
            S_n} \prod_{i=1}^n A_{i, \sigma(i)}$} matrike $P_f$.

            Za stohastične matrike velja spodnja lema.
            \begin{lema}
              Za stohastično matriko $P$ velja
              $\operatorname{Perm}(P)=1$ natanko tedaj, ko je $P$ permutacijska.
            \end{lema}
            \begin{dokaz}
              Če je $P$ permutacijska, ima očitno enotski permanent.
              Dokažimo lemo še v obratno smer.

              Naj bo $P =
              \begin{bmatrix}
                p_{i,j}
              \end{bmatrix}_{i,j=1,\dotsc, n}$ poljubna stohastična
              matrika s $\operatorname{Perm}(P) = 1$. Po definicijo
              so vsi elementi  $p_{i,j}$ matrike med $0$ in $1$,
              vrstice se pa seštejejo v $1$.
              Velja
              $$
              \operatorname{Perm}(P) = \sum_{\sigma  \in S_n}
              \prod_{i=1}^n p_{i, \sigma(i)} \leq \sum_{\sigma  \in
              \funnn{n}} \prod_{i=1}^n p_{i, \sigma(i)},
              $$
              saj so vsi elementi stohastične matrike nenegativni, in
              smo vsoti zgolj prišteli nenegativne elemente.

              Vsako funckijo $\sigma \in \funnn{n}$ lahko predstavimo
              z vektorjem $(\sigma_1, \sigma_1, \dotsc, \sigma_n)$,
              kjer je $\sigma_i = \sigma(i) \in \N$. Tako je
              $$
              \sum_{\sigma  \in \funnn{n}} \prod_{i=1}^n p_{i, \sigma(i)} =
              \sum_{(\sigma_1, \sigma_1, \dotsc, \sigma_n) \in [n]^n}
              \prod_{i=1}^n p_{i, \sigma_i} =
              \prod_{i=1}^n (p_{i, 1} + p_{i, 2} + \dotsm + p_{i, n}).
              $$
              Ker so vsote vrstic enake $1$, velja
              $$
              1 = \operatorname{Perm}(P) \leq  \sum_{\sigma  \in
              \funnn{n}} \prod_{i=1}^n p_{i, \sigma(i)} =
              \prod_{i=1}^n (p_{i, 1} + p_{i, 2} + \dotsm + p_{i, n}) = 1.
              $$
              Sledi, da je  $\sum_{\sigma  \in S_n}  \prod_{i=1}^n
              p_{i, \sigma(i)} =  \sum_{\sigma  \in \funnn{n}}
              \prod_{i=1}^n p_{i, \sigma(i)} $ in posledično je vsota
              \\
              $
              \sum_{\sigma  \in \funnn{n} \setminus S_n}
              \prod_{i=1}^n p_{i, \sigma(i)}
              $ enaka $0$.

              Zdaj lahko dokažemo, da je  v poljubnem stolpcu matrike
              $P$ lahko največ en ne-ničelni element. Naj bo $i$
              indeks poljubnega stolpca in $r \neq s$ indeksa dveh
              vrstic. Recimo, da sta elementa $p_{r, i}$ in $p_{s,i}$
              oba neničelna.

              Definiramo lahko tako preslikavo $f \in \funnn{n}$, da
              je $f(r) = f(s) = i$, za $j \notin \{r, s\}$ pa $f(j)$
              izberemo tako, da $p_{j, f(j)}$ ni ničelen (to lahko
                vedno izberemo, saj je matrika stohastična, torej ima v
              vsaki vrstici vsaj en neničelen element).

              Očitno je $\prod_{i=1}^n p_{i, f(i)} \neq 0$, kar pa je
              v protislovju z $ \displaystyle \sum_{\sigma  \in
              \funnn{n} \setminus S_n} \prod_{i=1}^n p_{i,
              \sigma(i)}$.   V vsakem stolpcu matrike je torej
              natanko en neničelen element.

              Velja še $1 = P(f \in S_n) = P(f \text{ je
              surjektivna}) = \prod_{i=1}^n P(\exists j \in \N. f(j)= i) =
              \prod_{i=1}^n (\sum_{j=1}^n p_{j, i})$, torej se vsote
              stolpcev zmnožijo v $1$. Ker je v vsakem stolpcu
              natanko en neničelen element sledi, da je ta element
              enak $1$.  Posledično je tudi v vsaki vrstici natanko
              en neničelen element (sicer se vrstica ne bi seštela v
              $1$), kar pomeni, da je $P$ permutacijska.
            \end{dokaz}
            Slučajna preslikava $f$ je torej skoraj zagotovo
            bijektivna natanko tedaj, ko je $P_f$ permutacijska.
            Enostavno je preveriti, da so stohastične matrike
            permutacijske natanko tedaj, ko so unitarne. Bolj, kot bo
            $P_f$ unitarna, bolj verjetno bo $f$ bijektivna.

            Če $P_f$ definiramo kot
            $$P_f = \operatorname{softmax}(\phi)$$(glej poglavje
            \ref{section:parametrizacija stohastičnih matrik}), lahko
            za funkcijo izgube vzamemo $\phi \mapsto \Loss{unitary}(P_f)$.

            \begin{definicija}
              Za poljubno realno matriko $\phi \in \R^{n \times n}$
              \emph{funkcijo izgube za bijektivnost} definiramo kot
              $$
              \Loss{bijective} = \Loss{unitary} \circ \operatorname{softmax}.
              $$
            \end{definicija}
            Z minimiziranjem iste funkcije izgube lahko rešujemo dva
            različna problema (problem iskanja unitarnih matrik in
            problem iskanja bijekcij) preslikava
            $\operatorname{softmax}$  pa pretvarja med problemoma.
            \subsection{Funkcija izgube za povezanost}
            Naj bosta $G_1 = ([n], E_1)$ in $G_2 = ([n], E_2)$ grafa
            in $M_1, M_2$ njuni matriki sosednosti.
            Za $i \sim_1 j$ bi radi, da je $P(f(i)  \sim_2 f(j)) = 1$.
            Računamo
            $$
            P(f(i) \sim_2 f(j)) = \sum_{k= 1}^n \sum_{h = 1}^n p_{i,
            k} p_{j, h} m_{k,h}^{(2)}
            % = f_j^T M_1 f_i
            = (P M_2 P^T)_{i, j}
            $$
            \begin{definicija}
              \emph{Funkcijo izgube za povezanost}\footnote{$\log$
              računamo po elementih}. lahko definiramo kot
              \begin{align*}
                \mathcal{L}_\sim &= -\sum_{i \sim_1 j} \log P(f(i) \sim_2 f(j))
                \\&= -\sum_{i=1}^n\sum_{j=1}^n log(f_j^T M_2
                f_i)m_{i,j}^{(1)}\\&=
                -\operatorname{tr}(\log(P_fM_2P_f^T) M_1^T).
              \end{align*}
            \end{definicija}

            \begin{opomba}
              \label{opomba:loss_function_qap}
              Ekvivalentno lahko funkcijo izgube definiramo kot
              $$
              \loss = ||M_1 - PM_2P^T ||_F^2.
              $$
              Slednja se uporablja pri \emph{graph matchingu} in je
              podrobneje opisana v poglavju \ref{section:graph_matching}.
            \end{opomba}
            \subsection{Rezultati}
            \label{subsection : rezultati_grafi_vanilla}
            \begin{primer}
              Poglejmo si preprost graf \ref{fig:primer_grafa} na
              petih točkah $\{0,1,2,3,4\}$, s povezavami $\{0 \sim 4,
              4 \sim 3, 3 \sim 0, 3 \sim 1, 2 \sim 0\}$. Očitno je
              grupa avtomorfizmov tega grafa izomorfna diedrski grupi
              trikotnika $D_3$.
              \begin{figure}[h!]
                \centering
                \includegraphics[width=0.5\linewidth]{images/random_samples_big_random_graph_1_on_5.png}
                \caption{Graf na petih vozljiščih.}
                \label{fig:primer_grafa}
              \end{figure}
              Gradientni spust poleg trivialne rešitve na določenih
              začetnih parametrih uspe najti avtomorfizem $(03)(24)$,
              na drugih pa konvergira do limite, ki ni blizu avtomorfizma.

            \end{primer}

            Rezultati simulacij nad različnim cayleyevim igrafi so v
            \ref{dodatek:cayley},
            nad različnimi naključno tvorjenimi grafi pa v
            \ref{dodatek:naključni grafi}.

            \subsection{Quadratic assigment \\ Graph matching \TODO{slo?}}
            \label{section:graph_matching}
            \begin{definicija}
              Naj bosta $M_1$ in $M_2$ matriki sosednosti dveh grafov
              na $n$ vozljiščih. \emph{Graph matching}
              \cite{lyzinski2015graphmatchingrelaxrisk} je problem
              iskanja minimuma
              $$
              \operatorname{argmin}_{P \in \Pi} ||M_1 - PM_2P^T ||_F^2,
              $$
              po  množici vseh $n \times n$ permutacijskih  matrik $\Pi$.

              Če matriki $M_1$ in $M_2$ pripadata izomorfnima
              grafoma, potem iščemo izomorfizem (glej opombo
              \ref{opomba:loss_function_qap}).
            \end{definicija}

            \emph{Graph matching} je NP-težek problem
            \cite{sahni1976qapisnphard}. Klasični pristopi problem
            rešujejo z relaksacijami funkcije izgube $||M_1 - P M_2
            P^T||_F^2$ nad večje družine matrik $P$ in prevedbo
            končne matrike nazaj na permutacijsko
            \cite{lyzinski2015graphmatchingrelaxrisk},
            \cite{aflalo2025convexrelaxation}.

            Novejše metode se problema lotevajo z globokim učenjem.
            \TODO{Ali dodam opise nekaterim metod?}
            %%%%%%%%%
            %   \subsubsection{REINFORCE}
            %   \TODO{Kaj je reinforce, kako se ga uporabi tukaj}
            %   \subsubsection{Relaksacija permutacijske matrike}
            %   \TODO{Kako to uporabljajo ,vir}
            %
            \subsection{Uporaba tabele inverzij}
            V poglavju \ref{subsection : rezultati_grafi_vanilla}
            izomorfizem iščemo v podprostoru
            distribucij nad $\funnn{n}$, ki ga parametriziramo s
            $(\R ^{n \times n})^{|S|}$. V tem poglavju predstavimo
            alternativen pristop,
            ki izomorfizem išče v podprostoru distribucij nad $S_n$. S tem
            se ognemo potrebi po uporabi funkcije izgube za bijektivnost.
            %
            %

            Množico permutacij predstavimo s tabelami inverzij. Nad
            slednjimi lahko na
            trivialen način definiramo gladko družino porazdelitev z
            $O(n^2)$ parametri.
            \subsubsection{Porazdelitve nad tabelo inverzij}
            \begin{definicija}(Tabela inverzij)
              Naj bo $\sigma \in S_n$ permutacija. Za vsak $k \in [n]$
              definiramo $a_k = |\{ j > k
                \mid \sigma(j) < \sigma(k)
              \}|$. Vektor
              $$\operatorname{TI}(\sigma) =
              (a_1,
                a_2, \dotsc,
              a_n)$$
              se
              imenuje \emph{tabela inverzij} permutacije $\sigma$.
            \end{definicija}
            Tabela inverzij $TI \colon S_n \to [0, n-1] \times [0,
            n-2 ] \times \dotsm \times [0]$
            je bijekcija \cite[Trditev
            1.~3.~12]{ernumerativecombinatorics2011volume1}.
            Porazdelitev vektorjev iz
            $\prod_{i=n-1}^0 [0, i]$ trivialno inducira porazdelitev nad $S_n$.

            Tabelo inverzij lahko razumemo kot proces tvorbe
            permutacij. Permutacijo $\sigma$ s tabelo inverzij
            $\operatorname{TI}(\sigma) = (a_1, a_2, \dotsc, a_n) \in
            \prod_{i=n-1}^0 [0, i]$ dobimo tako, da
            $1$ postavimo na $(a_1 +1 )$-to mesto v v tabeli $[1,2,\dotsc, n]$,
            nato $2$ postavimo na $(a_2 +1)$-to prosto mesto,
            in tako naprej, dokler ne postavimo $n$ na $(a_n +1)$-to
            prosto mesto.
            \begin{primer}
              \label{primer:tabela_inverzij}
              Naj bo  $\operatorname{TI}(\sigma) = (2, 1, 0)$.
              Začnemo s prazno tabelo
              $$[ \_, \_, \_]$$
              dolžine $3$. Postavimo $1$ na $3$-to mesto
              $$[ \_, \_, 1]$$
              nato $2$ na $3$-to prosto mesto
              $$[ \_, 2, 1]$$
              in na koncu $3$ na $2$-to prosto mesto
              $$[ 3, 2, 1].$$
              Velja $\sigma = (13)(2)$.
            \end{primer}

            Na podoben način lahko definiramo porazdelitev nad permutacijami.
            Naj bo
            $$P =
            \begin{bmatrix}
              p_{1,1} & p_{1,2} & \dotsm & p_{2, n-1} & p_{1,n} \\
              p_{2,1} & p_{2,2} & \dotsm & p_{2, n-1} & 0 \\
              p_{3,1} & p_{3,2} & \dotsm & 0 & 0 \\
              \vdots & \vdots & \iddots & \vdots & \vdots    \\
              p_{n-1,1} & p_{n-1,2}  & \dotsm & 0  & 0\\
              p_{n,1} & 0  & \dotsm & 0  & 0\\
            \end{bmatrix}$$
            stohastična matrika. Potem lahko definiramo
            $$P(a_k = i) = P(k\text{ postavimo na } i\text{-to prazno
            mesto}) = p_{k, i}$$
            in
            $$P(TI(\sigma) = (x_1, x_2, \dotsc, x_n)) = \prod_{i=1}^n
            P(a_k = x_k).$$
            \begin{primer}
              Naj bo
              $$
              P =
              \begin{bmatrix}
                a & b & c \\
                d & e & 0 \\
                f & 0 & 0 \\
              \end{bmatrix}
              $$
              stohastična matrika, ki podaja porazdelitev nad $S_3$.
              Verjetnost za permutacijo $\sigma = (13)(2)$ iz primera
              \ref{primer:tabela_inverzij} je
              $$P(TI(\sigma) = (2, 2, 1)) = P(a_1 = 2) P(a_2 = 2)
              P(a_3 = 1) = cef$$.
            \end{primer}
            %
            %
            % Množica vseh možnih porazdelitev nad $\prod_{i=n-1}^0 [0,
            % i]$ je ogromna.
            % Podobno kot v \TODO{delovanja} se omejimo na manjšo
            % parametrično družino porazdelitev
            % \begin{equation}
            %   \label{eq:porazdelitev nad tabelo inverzij}
            %   P(a_k = i) = \phi_{k,i}
            % \end{equation}
            % in definiramo
            % $$
            % P(a_k = i) = P(k\text{ postavimo na }i\text{-to prosto mesto})
            % $$
            %
            \subsubsection{Funkcija izgube}
            Naj bosta $M_1$ in $M_2$ matriki sosednosti dveh grafov
            na $n$ vozljiščih. Za povezani vozljišči $i \sim_1 j$ bi radi, da je
            $P(f(i) \sim_2 f(j)) = 1$, kjer je $f$ slučajna permutacija.

            Za funkcijo izgube lahko vzamemo
            $$
            \loss = \sum_{i, j} ( (M_1)_{i,j} - P(f(i) \sim f(j)) )^2.
            $$
            Očitno velja
            $$
            P(f(i) \sim f(j)) = \sum_{k=1}^n \sum_{h=1}^n P(f(i) = k,
            f(j) = h) (M_2)_{k,h}.
            $$
            \subsubsection{Hiter izračun funkcije izgube}
            Zanima nas $P(f(i) = k, f(j) = h)$.
            Najprej uvedemo pomožne funkcije:
            \begin{align*}
              S_m^<(i) &= \sum_{s=1}^{i} P(a_m = s) \\
              S_m^\text{mid}(i,j) &= \sum_{s=i}^{j} P(a_m = s) \\
              S_m^>(j) &= \sum_{s=j}^{n} P(a_m = s) \\
            \end{align*}
            Recimo, da smo že postavili $m-1$ elementov
            in postavljamo $m$-ti element. S
            $q(j,h,m)$ označimo verjetnost, da bomo med
            tvorjenjem slučajen premutacije $h$ postavili na
            $j$-to trenutno prosto mesto. Podobno s $p(i, k, j, h,
            m)$ označimo verjetnost, da bomo
            med tvorjenjem slučajne permutacije $k$ postavili na
            $i$-to in $h$ na $j$-to trenutno prosto mesto.
            Očitno je $P(f(i) = k, f(j) = h) = p(i, k, j, h,1)$.
            %

            Za $h \notin [m, n]$ ali $j \notin [1, n-m+1]$ je
            $q(j,h,m) = 0$, sicer pa
            %Za $m=h$ in $j \in [1, n-m+1]$ velja
            %$$q(j,h,m) = P(a_m = j)$$ sicer pa velja
            %$$q(j, h, m) =   q(j-1, h, m+1) S_m^<(j-1) +
            %q(j, h, m+1) S_m^>(j+1)$$.
            %
            $$
            q(j, h, m) =
            \begin{cases}
              P(a_m = j) & m = h \\
              q(j-1, h, m+1) S_m^<(j-1) +
              & \text{sicer} \\
              q(j, h, m+1) S_m^>(j+1)&\\
            \end{cases}
            $$

            Za $p$ velja $p(i, k, j, h, m) = 0$, če velja ena od $k=h$,
            $k \notin [m, n]$, $h \notin [m, n]$, $i \notin [1,
            n-m+1]$, $j \notin [1, n-m+1]$.
            Sicer pa je
            %Če je $m=k$ in $k,h \in [m,n]$ ter $i,j \in [1, n-m+1]$,
            %je
            %$$
            %p(i, k, j, h, m) = P(a_k = i) q(j-1, h, k+1).
            %$$
            %Podobno je za $m=h$ in $k,h \in [m,n]$ ter $i,j \in [1,
            %n-m+1]$
            %$$
            %p(i, k, j, h, m) =  P(a_h = j) q(i, k, h+1).
            %$$
            %Sicer pa velja rekurzivna zveza
            %\begin{align*}
            %  p(i, k, j, h, m) &= \\
            %  p(i-1, k, j-1, h, m+1) &S_m^<(i-1) +\\
            %  p(i, k, j-1, h, m+1) &S_m^{\text{mid}}(i+1, j-1) +\\
            %  p(i, k, j, h, m+1) &S_m^>(j+1).
            %\end{align*}
            $$
            p(i, k, j, h, m) =
            \begin{cases}
              P(a_k = i) q(j-1, h, k+1) & m = k \\
              P(a_h = j) q(i, k, h+1) & m = h \\
              p(i-1, k, j-1, h, m+1) S_m^<(i-1) +
              &  \text{sicer} \\
              p(i, k, j-1, h, m+1) S_m^{\text{mid}}(i+1, j-1) +
              &\\
              p(i, k, j, h, m+1) S_m^>(j+1)
              & \\
            \end{cases}
            $$
            %
            Z uporabo zgornjih rekurzivnih
            enačb lahko funkcijo izgube izračunamo v času
            $O(n^5)$. S pametnim preoblikovanjem funkcije izgube lahko
            časovno zahtevnost znižamo na $O(n^3)$.
            %
            \paragraph{Izračun v $O(n^3)$}
            %
            Računamo
            $\loss = \sum_{i,j} \left ((M_1)_{i,j} - \sum_{k,h}p(i, k, j,
            h, 1) (M_2)_{k,h} \right )^2$.
            Definiramo pomožne funkcije
            \begin{align*}
              U_{j, m, k} &= \sum_h q(j, h, m)(M_2)_{k, h} \\
              V_{i, m, h} &= \sum_k q(i, k, m)(M_2)_{k, h} \\
              T_{i, j, m} &= \sum_{k,h} p(i, k, j, h, m) (M_2)_{k,h}
            \end{align*}
            Očitno je $\loss = \sum_{i,j} \left ((M_1)_{i,j} - T_{i,
            j, 0} \right )^2$. Iz rekurzivnih zvez za $p$ in $q$
            sledi
            \begin{align*}
              T_{i, j, m} &={} &
              &+\sum_h  P(a_m = i) q(j-1, h, m+1) (M_2)_{m,h}  \\
              &   &
              &+\sum_h  P(a_m = j) q(i, k, m+1) (M_2)_{k,m} \\
              &   &
              &+ \sum_{k, h}
              p(i-1, k, j-1, h, m+1) S_m^<(i-1) (M_2)_{k,h} \\
              &   &
              &+\sum_{k, h}
              p(i, k, j-1, h, m+1) S_m^{\text{mid}}(i+1, j-1)(M_2)_{k,h} \\
              &   &
              &+\sum_{k, h}
              p(i, k, j, h, m+1) S_m^>(j+1) (M_2)_{k,h} \\
              &={} &
              &P(a_m=i)  U_{j-1, m+1, m} +
              P(a_m=j) V_{i, m+1, m} \\
              & & &+
              T_{i-1, j-1, m+1}  S_m^<(i-1) +
              T_{i, j-1, m+1}  S_m^{\text{mid}}(i+1, j-1) +
              T_{i, j, m+1}  S_m^>(j+1)
            \end{align*}
            Z uporabo teh formul lahko funkcijo izgube izračunamo v
            času $O(n^3)$. Če želimo še hitrejši izračun, lahko
            naključno izberemo podmnožico indeksov $(i,j)$ in
            izračunamo stohastični približek funkcije izgube na podmnožici.

            Naj bo $S \subset [n] \times [n]$ naključno izbrana
            podmnožica indeksov. Za približek funkcije izgube lahko
            vzamemo
            \begin{equation*}
              \loss_\approx = \sum_{(i, j) \in S}
              \left ((M_1)_{i,j} - T_{i, j, 1} \right )^2.
            \end{equation*}
            Časovna zahtevnost izračuna približka je
            $O(|S| n^2)$.
            %
            \subsubsection{Overparametrizacija z nevronskimi mrežami}
            Verjetnost uspešne konvergence lahko poizkusimo
            izboljšati s overparametrizacijo proglema.
            Matriko $P$ definiramo kot izhod globoke nevronske mreže
            (glej dodatek \ref{poglavje:nn}) in spreminjamo njene
            parametre.

            Overparametrizacija z nevronskimi mrežami deluje za
            \emph{gradient dominante}
            funkcije izgube
            \cite{allenzhu2019convergencetheorydeeplearning}, glej
            \ref{poglavje:konvergenca-globokih-mrez}. Za splošne
            funkcije izgube obstajajo
            empirični rezultati, ki kažejo, da z rahlo
            overparametrizacijo in gradientnim spustom lahko dosežemo
            uspešno konvergenco
            \cite{safran2021effectsmildoverparameterizationoptimization,
            simon2024bettermodernmachinelearning}.
            .
            \TODO{deep network --> malo minimumov, veliko sedlov , kar je gut}
            \subsection{Rezultati}
            \TODO{}

            \section{Overparametrizacija}

            \section*{Dodatek: Konvergenca globokih mrež}
            \addcontentsline{toc}{section}{Dodatek: Konvergenca globokih mrež}
            \label{poglavje:konvergenca-globokih-mrez}
            \TODO{Sem gredo definicije mrež in }

            \clearpage
            \section{Preparametrizacije}
            \label{section:overparametrisation}
            Metode, predstavljene v prejšnjih poglavjih, so zanimive iz
            teoretičnega stališča, a niso uporabne v praksi,
            saj je dinamika gradientnih tokov preveč kaotična in
            verjetnost za neuspešno konvergenco previsoka.

            Aplikativnosti se lahko približamo z
            \emph{overparametrzicaijo}. Stohastično matriko $P$ lahko
            definiramo z večjim modelom $P(\phi)$, ki je odvisen od
            velikega števila parametrov $\phi \in \R^d, d \gg n^2$.
            V
            \subsection{Nevronske mreže}
            \label{poglavje:nn}
            Matriko $P(\phi)$ bomo modelirali z \emph{nevronskimi
            mrežami}. Definicija nevronske mreže je povzeta po
            \cite{prince2023understandingdeeplearning}.
            \begin{definicija}[Nevronska mreža]
              Naj bodo $W_1 \colon \R^{d_0} \to \R^{d_1}, W_2 \colon
              \R^{d_1} \to \R^{d_2}, \dotsc , W_l \colon \R^{d_{l-1}}
              \to \R^{d_l}$ \emph{afine} preslikave. Za vsak $i \in
              \{1, 2, \dotsc, l\}$ in $x \in R^{d_{i-1}}$ velja
              $W_i(x) = \Omega_i x + b_i$ za $\Omega_i \in
              \R^{d_{i-1} \times d_i}$ in $b_i \in \R^{d_{i}}$.

              Naj bodo $\sigma_1, \dotsc, \sigma_l$ realne funkcije v
              eni spremenljivki. Za poljuben $d \in \N$ definiramo
              $\sigma_i \colon \R^d \to \R^d$ po elementih s
              predpisom $\sigma(x)_i = \sigma(x_i)$. Preslikavam
              $\sigma_i$ rečemo \emph{aktivatorske funkcije}.
              Običajno aktivacijske funkcije izberemo izmed
              $\operatorname{ReLU}\footnote{$\operatorname{ReLU}(x) =
                \begin{cases}
                  0 & x \leq 0 \\
                  x & x > 0
                \end{cases}
              $}, \operatorname{softmax}, \arctan, \operatorname{GeLU}$.

              \emph{Nevronska mreža} globine $l$ je model
              $$
              M_\phi = \sigma_l \circ W_l \circ \sigma_{l-1} \circ
              W_{l-1} \circ \dotsm \circ \sigma_1 \circ W_1 \colon
              \R^{d_0} \to \R^{d_l},
              $$
              odvisen od parametrov $\phi  = \Omega_0, b_0, \Omega_1,
              b_1, \dotsc, \Omega_l, b_l$. Če je $l \geq$ rečemo, da
              je mreža globoka.
            \end{definicija}
            \begin{definicija}[Regresija]
              Eden od klasičnih problemov, pri katerem se uporabljajo
              nevronske mreže, je \emph{regresija}.

              Za slučajni spremenljivki $X \in \R^{d_1}$ in $Y \in
              \R^{d_2}$ iščemo model $\hat Y (X)$, ki čim bolje opiše $E[Y|X]$.

              Predpostavimo, da imamo množico vzorcev $S_\text{train}
              = \{(x_i, y_i) \mid i \in [n]\}$ slučajne spremenljivke
              $(X,Y)$ in nevronsko mrežo $M_{\phi_0} \colon \R^{d_0}
              \to \R^{d_l}$ z naključno izbranimi začetnimi parametri
              $\phi$ in z gradientnim spustom minimaliziramo
              $$
              \Loss{regresija} (\phi) = \frac{1}{|S_\text{train}
              |}\sum _{(x, y) \in S_\text{train}} ||M_\phi (x) - y  ||^2.
              $$
              Uspešnost modela nato testiramo na množici vzorcev
              $S_\text{test} \subset S_\text{train}^c$.
            \end{definicija}
            %
            \begin{definicija}[Grafovska nevronska mreža]
              \TODO{}
            \end{definicija}

            \subsection{Nevronske mreže dosežejo globalni minimum
            funckije izgube}
            Izkaže se, da z večanjem nevronske mreže $M(\phi)$
            poljubna gladka funkcija izgube $\loss (\phi) $ postaja vse
            bolj konveksna. V praksi to pomeni, da za velike
            nevronske mreže gradientni spust najde globalni minimum.
            Spodnji izrek je povzet po
            \cite{allenzhu2019convergencetheorydeeplearning}.
            \begin{izrek}[Allen-Zhu, Li, Song]
              \label{izrek:allen-zhu-global-minima}
              Naj bo $\loss \colon \R^d \to [0, \infty)$ gladka \TODO{}

              Naj bo $S_\text{train}$ taka množica vzorcev, da
              obstaja $\delta > 0$, da za dva  vzorca $(x_i, y_i),
              (x_j,y_j) \in S_\text{train}$ z različnima indeksoma $i
              \neq j $ velja $||x_i - x_j|| > \delta$.

              Naj bo $M = \operatorname{ReLU} \circ W_l \circ
              \operatorname{ReLU} \circ W_{l-1} \circ \dotsm \circ
              \operatorname{ReLU} \circ W_1$  taka $l$-plastna
              nevronska mreža, da velja $d_1 = d_2 = \dotsm = d_{l-1}
              = m \in \N > d_0 \cdot \operatorname{poly}(n, l, \delta^{-1})$.

              Potem gradinetni spust s hitrostjo učenja $\eta =
              \Theta (d_0 \frac{\delta}{m \cdot
              \operatorname{poly}(n, l)})$ z verjetnostjo vsaj $1 -
              e^{- \Omega(\log^2 m)}$  po
              $\Theta(\operatorname{poly}(n, l) \delta^{-2} \epsilon
              ^{-1})$ korakih najde parametre $\phi$, za katere velja
              $\Loss{regresija}(\phi) < \epsilon$.
            \end{izrek}
            Dovolj velika nevronska mreža s preprosto obliko lahko
            poljubno dobro aproksimira povprečno kvadratno napako pri
            regresiji. Podoben izrek velja za poljubno gladko funkcijo izgube.

            Z overparametrizacijo se lahko znebimo konvergence do
            lokalnih minimumov in z gradientnim spustom poljubno
            verjetno konvergiramo do globalne rešitve. Iskano
            stohastično matriko $P(\phi)$ lahko definiramo z
            nevronsko mrežo, izračunano na nekem konstantnem vektorju
            $x_0 \in \R^{d_0}$.
            \TODO{Rajši daj že splošen rezultat za poljuben loss.
            Glej stran 38 v članku}
            % Vir: \cite{allenzhu2019convergencetheorydeeplearning}
            \TODO{Rezultati}
            \section{Dodatek}
            \subsection{Izomorfizmi Cayleyevih grafov}
            \label{dodatek:cayley}
            \TODO{}
            % \includepdf[pages=-]{pdfs/aut_calyey.pdf}

            \subsection{Izomorfizmi naključnih grafov}
            \label{dodatek:naključni grafi}
            \TODO{Sem gredo dodatki, dodatni doakzi, itd..}
            %            \subsection{Dovolj so upodobitve nad
            % končnimi množicami}
            %
            % \section{Integrali po \texorpdfstring{$\omega$}{ω}-kompleksih}
            % \subsection{Definicija}
            % \begin{definicija}
            %   Neskončno zaporedje kompleksnih števil, označeno z
            % $\omega = (\omega_1, \omega_2, \ldots)$,
            %   se imenuje \emph{$\omega$-kompleks}.\footnote{To ime
            % je izmišljeno.}

            %   Črni blok zgoraj je tam namenoma. Označuje, da
            % \LaTeX{} ni znal vrstice prelomiti pravilno
            %   in vas na to opozarja. Preoblikujte stavek ali mu
            % pomagajte deliti problematično besedo z
            %   ukazom \verb|\hyphenation{an-ti-ko-mu-ta-ti-ven}| v preambuli.
            % \end{definicija}
            % \begin{trditev}[Znano ime ali avtor]
            %   \label{trd:obstoj-omega}
            %   Obstaja vsaj en $\omega$-kompleks.
            % \end{trditev}
            % \begin{proof}
            %   Naštejmo nekaj primerov:
            %   \begin{align}
            %     \omega &= (0, 0, 0, \dots), \label{eq:zero-kompleks} \\
            %     \omega &= (1, i, -1, -i, 1, \ldots), \nonumber \\
            %     \omega &= (0, 1, 2, 3, \ldots). \nonumber \qedhere
            % % postavi QED na zadnjo vrstico enačbe
            %   \end{align}
            % \end{proof}

            % \section{Tehnični napotki za pisanje}

            % \subsection{Sklicevanje in citiranje}
            % Za sklice uporabljamo \verb|\ref|, za sklice na enačbe
            % \verb|\eqref|, za citate \verb|\cite|. Pri
            % sklicevanju in citiranju sklicano številko povežemo s
            % prejšnjo besedo z nedeljivim presledkom
            % $\sim$, kot
            % npr.\ \verb|iz trditve~\ref{trd:obstoj-omega} vidimo|.

            % \begin{primer}
            %   Zaporedje~\eqref{eq:zero-kompleks} iz dokaza
            % trditve~\ref{trd:obstoj-omega} na
            %   strani~\pageref{trd:obstoj-omega} lahko najdemo tudi
            % v Spletni enciklopediji zaporedij~\cite{oeis}.
            %   Citiramo lahko tudi bolj natančno~\cite[trditev 2.1,
            % str.\ 23]{lebedev2009introduction}.
            % \end{primer}

            % \subsection{Okrajšave}
            % Pri uporabi okrajšav \LaTeX{} za piko vstavi predolg
            % presledek, kot npr. tukaj. Zato se za vsako
            % piko, ki ni konec stavka doda presledek običajne širine
            % z ukazom \verb*|\ |, kot npr.\ tukaj.
            % Primerjaj z okrajšavo zgoraj za razliko.

            % \subsection{Vstavljanje slik}
            % Sliko vstavimo v plavajočem okolju \texttt{figure}.
            % Plavajoča okolja \emph{plavajo} po tekstu, in
            % jih lahko postavimo na vrh strani z opcijskim
            % parametrom `\texttt{t}', na lokacijo, kjer je v kodi s
            % `\texttt{h}', in če to ne deluje, potem pa lahko rečete
            % \LaTeX u, da ga \emph{res} želite tukaj,
            % kjer ste napisali, s `\texttt{h!}'. Lepo je da so
            % vstavljene slike vektorske (recimo \texttt{.pdf}
            % ali \texttt{.eps} ali \texttt{.svg}) ali pa
            % \texttt{.png} visoke resolucije (več kot
            % \unit[300]{dpi}).  Pod vsako sliko je napis in na vsako
            % sliko se skličemo v besedilu. Primer
            % vektorske slike je na sliki~\ref{fig:sample}. Vektorsko
            % sliko prepoznate tako, da močno
            % zoomate v sliko, in še vedno ostane gladka. Več
            % informacij je na voljo na
            % \url{https://en.wikibooks.org/wiki/LaTeX/Floats,_Figures_and_Captions}.
            % Če so slike bitne, kot na
            % primer slika~\ref{fig:image}, poskrbite, da so v dovolj
            % visoki resoluciji.

            % \begin{figure}[h]
            %   \centering
            %   \includegraphics[width=0.6\textwidth]{images/sample.pdf}
            % % \caption[caption za v kazalo]{Dolg caption pod sliko}
            %   \caption[Primer vektorske slike.]{Primer vektorske
            % slike z oznakami v enaki pisavi, kot jo
            %      uporablja \LaTeX{}.  Narejena je s programom
            % Inkscape, \LaTeX{} oznake so importane v
            %      Inkscape iz pomožnega PDF.}
            %   \label{fig:sample}
            % \end{figure}

            % \begin{figure}[h]
            %   \centering
            %   \includegraphics[width=0.8\textwidth]{images/image.png}
            %   \caption[Primer bitne slike.]{Primer bitne slike,
            % izvožene iz Matlaba. Poskrbite, da so slike v
            %   dovolj visoki resoluciji in da ne vsebujejo prosojnih
            % elementov (to zahteva PDF/A-1b format).}
            %   \label{fig:image}
            % \end{figure}

            % \subsection{Kako narediti stvarno kazalo}
            % Dodate ukaze \verb|\index{polje}| na besede, kjer je
            % pojavijo, kot tukaj\index{tukaj}.
            % Več o stvarnih kazalih je na voljo na
            % \url{https://en.wikibooks.org/wiki/LaTeX/Indexing}.

            % \subsection{Navajanje literature}
            % Članke citiramo z uporabo \verb|\cite{label}|,
            % \verb|\cite[text]{label}| ali pa več naenkrat s
            % \verb|\cite\{label1, label2}|. Tudi tukaj predhodno
            % besedo in citat povežemo z nedeljivim presledkom
            % $\sim$. Na primer~\cite{chen2006meshless,liu2001point},
            % ali pa \cite{kibriya2007empirical}, ali pa
            % \cite[str.\ 12]{trobec2015parallel}, \cite[enačba
            % (2.3)]{pereira2016convergence}.
            % Vnosi iz \verb|.bib| datoteke, ki niso citirani, se ne
            % prikažejo v seznamu literature, zato jih
            % tukaj citiram.~\cite{vene2000categorical},
            % \cite{gregoric2017stopniceni}, \cite{slak2015induktivni},
            % \cite{nsphere}, \cite{kearsley1975linearly},
            % \cite{STtemplate}, \cite{NunbergerTand},
            % \cite{vanoosten2008realizability}.

            \end{document}
